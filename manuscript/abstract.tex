\begin{abstract}

%%% Leave the Abstract empty if your article does not require one, please see the Summary Table for full details.
\section{}
Equivalent-layer technique is a powerful tool for processing potential-field data in the space domain. However, the greatest hindrance for using the equivalent-layer technique is its high computational cost for processing massive data sets. The large amount of computer memory usage to store the full sensitivity matrix combined with the computational time required for matrix-vector multiplications and to solve the resulting linear system, are the main drawbacks that made unfeasible the use of the equivalent-layer technique for a long time. More recently, the advances in computational power propelled the development of methods to overcome the heavy computational cost associated with the equivalent-layer technique. We present a comprehensive review of the computation aspects concerning the equivalent-layer technique addressing how previous works have been dealt with the computational cost of this technique. Historically, the high computational cost of the equivalent-layer technique has been overcome by using a variety of strategies such as: moving data-window scheme, equivalent data concept, wavelet compression, quadtree discretization, reparametrization of the equivalent layer by a piecewise-polynomial function, iterative scheme without solving a system of linear equations and the convolutional equivalent layer using the concept of block-Toeplitz Toeplitz-block (BTTB) matrices. We compute the number of floating-point operations of some of these strategies adopted in the equivalent layer technique to show their effectiveness in reducing the computational demand. Numerically, we also address the stability of some of these strategies used in the equivalent layer technique by comparing with the stability via the classic equivalent-layer technique with the zeroth-order Tikhonov regularization.

\tiny
 \keyFont{ \section{Keywords:} equivalent layer, gravimetry, fast algorithms, computational cost, stability analysis} %All article types: you may provide up to 8 keywords; at least 5 are mandatory.
\end{abstract}