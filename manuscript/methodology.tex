\section{Fundamentals}

Let $\mathbf{d}$ be a $D \times 1$ vector, whose $i$-th element $d_{i}$ is the observed potential
field at the position $(x_{i}, y_{i}, z_{i})$, $i \in \{1:D\}$, of a topocentric Cartesian system with
$x$, $y$ and $z$ axes pointing to north, east and down, respectively.
Consider that $d_{i}$ can be satisfactorily approximated by a harmonic function
\begin{equation}
	f_{i} = \sum\limits_{j = 1}^{P} g_{ij} \, p_{j} \: ,
	\quad i \in \{1:D\} \: ,
	\label{eq:predicted-data-f-i}
\end{equation}
where, $p_{j}$ represents the scalar physical property of a virtual source (i.e., monopole, dipole, prism) located
at $(x_{j}, y_{j}, z_{j})$, $j \in \{1:P\}$ and 
\begin{equation}
	g_{ij} \equiv g(x_{i} - x_{j}, y_{i} - y_{j}, z_{i} - z_{j}) \: ,
	\quad z_{i} < \min\{z_{j}\} \: , \quad \forall i \in \{1:D\} \: ,
	\label{eq:harmonic-function-g-ij}
\end{equation}
is a harmonic function, where $\min\{z_{j}\}$ denotes the minimum $z_{j}$, or the vertical coordinate of the 
shallowest virtual source.
These virtual sources are called \textit{equivalent sources} and they form an \textit{equivalent layer}.
In matrix notation, the potential field produced by all equivalent sources at all points 
$(x_{i}, y_{i}, z_{i})$, $i \in \{1:D\}$, is given by:
\begin{equation}
	\mathbf{f} = \mathbf{G} \mathbf{p} \: ,
	\label{eq:predicted-data-vector}
\end{equation}
where $\mathbf{p}$ is a $P \times 1$ vector with $j$-th element $p_{j}$ representing the scalar physical property
of the $j$-th equivalent source 
and $\mathbf{G}$ is a $D \times P$ matrix with element $g_{ij}$ given by equation \ref{eq:harmonic-function-g-ij}. 

The equivalent-layer technique consists in solving a linear inverse problem to determine a parameter vector $\mathbf{p}$ 
leading to a predicted data vector $\mathbf{f}$ (equation \ref{eq:predicted-data-vector}) \textit{sufficiently close to} the 
observed data vector $\mathbf{d}$, whose $i$-th element $d_{i}$ is the observed potential field at $(x_{i}, y_{i}, z_{i})$.
The notion of \textit{closeness} is intrinsically related to the concept of \textit{vector norm} \cite[e.g.,][p. 68]{golub-vanloan2013}
or \textit{measure of length} \cite[e.g.,][p. 41]{menke2018}.
Because of that, almost all methods for determining $\mathbf{p}$ actually estimate a parameter 
vector $\tilde{\mathbf{p}}$ minimizing a length measure of the difference between $\mathbf{f}$ and $\mathbf{d}$
(see subsection \ref{subsec:general-formulation}).
Given an estimate $\tilde{\mathbf{p}}$, it is then possible to compute a potential field transformation 
\begin{equation}
	\mathbf{t} = \mathbf{A} \tilde{\mathbf{p}} \: ,
	\label{eq:transformation}
\end{equation}
where $\mathbf{t}$ is a $T \times 1$ vector with $k$-th element $t_{k}$ representing the transformed potential field at
the position $(x_{k}, y_{y}, z_{k})$, $k \in \{1:T\}$, and
\begin{equation}
	a_{kj} \equiv a(x_{k} - x_{j}, y_{k} - y_{j}, z_{k} - z_{j}) \: ,
	\quad z_{k} < \min\{z_{j}\} \: , \quad \forall k \in \{1:T\} \: ,
	\label{eq:harmonic-function-a-kj}
\end{equation}
is a harmonic function representing the $kj$-th element of the $T \times P$ matrix $\mathbf{A}$.

\subsection{Spatial distribution and total number of equivalent sources}
\label{subsec:spatial-distribution-sources}

There is no well-established criteria to define the optimum number $P$ or the spatial distribution
of the equivalent sources. We know that setting an equivalent layer with more (less) sources than potential-field 
data usually leads to an underdetermined (overdetermined) inverse problem \cite[e.g.,][ p. 52--53]{menke2018}.
Concerning the spatial distribution of the equivalent sources, the only condition is that they must rely on a 
surface that is located below and does not cross that containing the potential field data.
\citet{soler-uieda2021} present a practical discussion about this topic.

From a theoretical point of view, the equivalent layer reproducing a given potential field data set cannot cross the
true gravity or magnetic sources. This condition is a consequence of recognizing that the equivalent layer is essentially an indirect solution of 
a boundary value problem of potential theory \citep[e.g.,][]{roy1962,zidarov1965,dampney1969,li_etal_2014,reis-etal2020}.
In practical applications, however, there is no guarantee that this condition is satisfied. 
Actually, its is widely known from practical experience \cite[e.g.,][]{gonzalez-etal2022} that the equivalent-layer technique
works even for the case in which the layer cross the true sources. 

Regarding the depth of the equivalent layer, \citet{dampney1969}  proposed a criterion based on horizontal data sampling, 
suggesting that the equivalent-layer depth should be between two and six times the horizontal grid spacing, 
considering evenly spaced data. 
However, when dealing with a survey pattern that has unevenly spaced data, \citet{reis-etal2020} adopted an alternative 
empirical criterion. 
According to their proposal, the depth of the equivalent layer should range from two to three times the spacing between 
adjacent flight lines.
The criteria of \citet{dampney1969} and \citet{reis-etal2020} are valid for planar equivalent layers. 
\citet{cordell1992} have proposed and an alternative criterion for scattered data that leads to an undulating equivalent layer.
This criterion have been slightly modified by \citet{guspi-etal2004}, \citet{guspi-novara2009} and \citet{soler-uieda2021},
for example, and consists in setting one equivalent source below each datum at a depth 
proportional to the horizontal distance to the nearest neighboring data points.
\citet{soler-uieda2021} have compared different strategies for defining the equivalent sources depth for the specific
problem of interpolating gravity data, but they have not found significant differences between them.


\subsection{Matrix $\mathbf{G}$}
\label{subsec:sensitivity-matrix}

Generally, the harmonic function $g_{ij}$ (equation \ref{eq:harmonic-function-g-ij}) is defined in terms of the 
inverse distance between the observation point $(x_{i}, y_{i}, z_{i})$ and the $j$-th equivalent source at $(x_{j}, y_{j}, z_{j})$,
\begin{equation}
	\frac{1}{r_{ij}} \equiv \frac{1}{\sqrt{(x_{i} - x_{j})^{2} + (y_{i} - y_{j})^{2} + (z_{i} - z_{j})^{2}}} \: ,
	\label{eq:inverse-distance-ij}
\end{equation}
or by its partial derivatives of first and second orders, respectively given by
\begin{equation}
	\partial_{\alpha} \frac{1}{r_{ij}} \equiv \frac{-(\alpha_{i} - \alpha_{j})}{r_{ij}^{3}} \: ,
	\quad \alpha \in \{ x, y, z \} \: ,
	\label{eq:deriv-1-inverse-distance-ij}
\end{equation}
and
\begin{equation}
	\partial_{\alpha\beta} \frac{1}{r_{ij}} \equiv 
	\begin{cases}
		\frac{3 \, (\alpha_{i} - \alpha_{j})^{2}}{r_{ij}^{5}} \: , &\alpha = \beta \: , \\
		\frac{3 \, (\alpha_{i} - \alpha_{j}) \, (\beta_{i} - \beta_{j})}{r_{ij}^{5}} - \frac{1}{r_{ij}^{3}} \: , &\alpha \ne \beta \: , \\
	\end{cases}
	\quad \alpha, \beta \in \{ x, y, z \} \: .
	\label{eq:deriv-2-inverse-distance-ij}
\end{equation}
In this case, the equivalent layer is formed by punctual sources representing monopoles or dipoles
\cite[e.g.,][]{dampney1969, emilia1973, leao-silva1989, cordell1992, oliveirajr-etal2013, siqueira-etal2017, reis-etal2020, takahashi-etal2020, soler-uieda2021, takahashi-etal2022}.
Another common approach consists in not defining $g_{ij}$ by using equations \ref{eq:inverse-distance-ij}--\ref{eq:deriv-2-inverse-distance-ij},
but other harmonic functions obtained by integrating them over the volume of regular prisms 
\cite[e.g.,][]{li-oldenburg2010, barnes-lumley2011, li_etal_2014, jirigalatu-ebbing2019}.
There are also some less common approaches defining the harmonic function $g_{ij}$ (equation \ref{eq:harmonic-function-g-ij})
as the potential field due to plane faces with constant physical property \citep{hansen-miyazaki1984}, doublets \citep{silva1986} or
by computing the double integration of the inverse distance function with respect to $z$ \citep{guspi-novara2009}.

A common assumption for most of the equivalent-layer methods is that the harmonic function $g_{ij}$ 
(equation \ref{eq:harmonic-function-g-ij}) is independent on the actual physical relationship between the
observed potential field and their true sources \cite[e.g.,][]{cordell1992, guspi-novara2009,li_etal_2014}.
Hence, $g_{ij}$ can be defined according to the problem.
The only condition imposed to this function is that it decays to zero as the observation point $(x_{i}, y_{i}, z_{i})$
goes away from the position $(x_{j}, y_{j}, z_{j})$ of the $j$-th equivalent source.
However, several methods use a function $g_{ij}$ that preserves the physical relationship between the
observed potential field and their true sources.
For the case in which the observed potential field is gravity data, $g_{ij}$ is commonly defined as a component of 
the gravitational field produced at $(x_{i}, y_{i}, z_{i})$ by a point mass or prism located at $(x_{j}, y_{j}, z_{j})$, with unit density.
On the other hand, $g_{ij}$ is commonly defined as a component of the 
magnetic induction field produced at $(x_{i}, y_{i}, z_{i})$ by a dipole or prism located at $(x_{j}, y_{j}, z_{j})$,
with unit magnetization intensity, when the observed potential field is magnetic data.

The main challenge in the equivalent-layer technique is the computational complexity associated with handling large datasets. 
This complexity arises because the sensitivity matrix $\mathbf{G}$ (equation \ref{eq:predicted-data-vector}) is dense regardless of the 
harmonic function $g_{ij}$ (equation \ref{eq:harmonic-function-g-ij}) employed. 
In the case of scattered potential-field data, the structure of $\mathbf{G}$  is not well-defined, regardless of the spatial distribution 
of the equivalent sources.
However, in a specific scenario where (i) each potential-field datum is directly associated with a single equivalent source located directly below it, 
and (ii) both the data and sources are based on planar and regularly spaced grids,  \citet{takahashi-etal2020,takahashi-etal2022} demonstrate that 
$\mathbf{G}$ exhibits a block-Toeplitz Toeplitz-block (BTTB) structure. 
In such cases, the product of $\mathbf{G}$ and an arbitrary vector can be efficiently computed using a 2D fast Fourier transform as a discrete convolution.

\section{Linear inverse problem of equivalent-layer technique}
\label{sec:linear-inverse-problem}

\subsection{General formulation}
\label{subsec:general-formulation}

A general formulation for almost all equivalent-layer methods can be achieved by first considering 
that the $P \times 1$ parameter vector $\mathbf{p}$ (equation \ref{eq:predicted-data-vector}) can be reparameterized 
into a $Q \times 1$ vector $\mathbf{q}$ according to:
\begin{equation}
	\mathbf{p} = \mathbf{H} \, \mathbf{q} \: ,
	\label{eq:reparameterization}
\end{equation}
where $\mathbf{H}$ is a $P \times Q$ matrix.
The predicted data vector $\mathbf{f}$ (equation \ref{eq:predicted-data-vector}) can then be
rewritten as follows:
\begin{equation}
	\mathbf{f} = \mathbf{G} \, \mathbf{H} \, \mathbf{q} \: .
	\label{eq:predicted-data-vetor-reparameterized}
\end{equation}
Note that the original parameter vector $\mathbf{p}$ is defined in a $P$-dimensional space whereas the reparameterized
parameter vector $\mathbf{q}$ (equation \ref{eq:reparameterization}) lies in a $Q$-dimensional space.
For convenience, we use the terms $P$-space and $Q$-space to designate them.

In this case, the problem of estimating a parameter vector $\tilde{\mathbf{p}}$ minimizing a length 
measure of the difference between $\mathbf{f}$ (equation \ref{eq:predicted-data-vector}) and $\mathbf{d}$
is replaced by that of estimating an auxiliary vector $\tilde{\mathbf{q}}$ minimizing the goal function
\begin{equation}
	\Gamma(\mathbf{q}) = \Phi(\mathbf{q}) + \mu \: \Theta(\mathbf{q}) \: ,
	\label{eq:function-Gamma}
\end{equation}
which is a combination of particular measures of length given by
\begin{equation}
	\Phi(\mathbf{q}) = \left( \mathbf{d} - \mathbf{f} \right)^{\top}\mathbf{W}_{d}\left( \mathbf{d} - \mathbf{f} \right) \: ,
	\label{eq:function-Phi}
\end{equation}
and
\begin{equation}
	\Theta(\mathbf{q}) = \left( \mathbf{q} - \bar{\mathbf{q}} \right)^{\top}\mathbf{W}_{q}\left( \mathbf{q} - \bar{\mathbf{q}} \right) \: ,
	\label{eq:function-Theta}
\end{equation}
where the regularization parameter $\mu$ is a positive scalar controlling the trade-off between the data-misfit function 
$\Phi(\mathbf{q})$ and the regularization function $\Theta(\mathbf{q})$; 
$\mathbf{W}_{d}$ is a $D \times D$ symmetric matrix defining the relative importance of each observed datum $d_{i}$;
$\mathbf{W}_{q}$ is a $Q \times Q$ symmetric matrix imposing prior information on $\mathbf{q}$;
and $\bar{\mathbf{q}}$ is a $Q \times 1$ vector of reference values for $\mathbf{q}$ that satisfies
\begin{equation}
	\bar{\mathbf{p}} = \mathbf{H} \, \bar{\mathbf{q}} \: ,
	\label{eq:reparameterization-reference}
\end{equation}
where $\bar{\mathbf{p}}$ is a $P \times 1$ vector containing reference values
for the original parameter vector $\mathbf{p}$.

After obtaining an estimate $\tilde{\mathbf{q}}$ for the reparameterized parameter vector $\mathbf{q}$ (equation \ref{eq:reparameterization}), 
the estimate $\tilde{\mathbf{p}}$ for the original parameter vector 
(equation \ref{eq:predicted-data-vector}) is computed by 
\begin{equation}
	\tilde{\mathbf{p}} = \mathbf{H} \, \tilde{\mathbf{q}} \: .
	\label{eq:vector-p-tilde}
\end{equation}

The reparameterized vector $\tilde{\mathbf{q}}$ is obtained by first computing the gradient of $\Gamma(\mathbf{q})$,
\begin{equation}
	\boldsymbol{\nabla} \Gamma(\mathbf{q}) = 
	-2 \, \mathbf{H}^{\top}\mathbf{G}^{\top} \mathbf{W}_{d} \left(\mathbf{d} - \mathbf{f} \right) +
	2 \, \mu \, \mathbf{W}_{q} \left( \mathbf{q} - \bar{\mathbf{q}} \right) \: .
	\label{eq:gradient-Gamma}
\end{equation}
Then, by considering that $\boldsymbol{\nabla} \Gamma(\tilde{\mathbf{q}}) = \mathbf{0}$ (equation \ref{eq:gradient-Gamma}),
where $\mathbf{0}$ is a vector of zeros, as well as adding and subtracting the term
$\left( \mathbf{H}^{\top}\mathbf{G}^{\top}\mathbf{W}_{d} \mathbf{G} \, \mathbf{H} \right) \bar{\mathbf{q}}$ ,
we obtain
\begin{equation}
	\tilde{\boldsymbol{\delta}}_{q} = \mathbf{B} \, \boldsymbol{\delta}_{d} \: ,
	\label{eq:vector-q-tilde}
\end{equation}
where 
\begin{equation}
	\tilde{\mathbf{q}} = \tilde{\boldsymbol{\delta}}_{q} + \bar{\mathbf{q}} \: ,
	\label{eq:delta-q-tilde}
\end{equation}
\begin{equation}
	\boldsymbol{\delta}_{d} = \mathbf{d} - \mathbf{G} \, \mathbf{H} \, \bar{\mathbf{q}} \: ,
	\label{eq:delta-d}
\end{equation}
\begin{equation}
	\mathbf{B} = \left( \mathbf{H}^{\top} \mathbf{G}^{\top} \mathbf{W}_{d} \, \mathbf{G} \, \mathbf{H} + 
	\mu \, \mathbf{W}_{q} \right)^{-1}
	\mathbf{H}^{\top} \mathbf{G}^{\top} \mathbf{W}_{d} \: ,
	\label{eq:matrix-B-overdetermined}
\end{equation}
or, equivalently \cite[][p. 62]{menke2018},
\begin{equation}
	\mathbf{B} = \mathbf{W}_{q}^{-1} \, \mathbf{H}^{\top} \mathbf{G}^{\top}
	\left( \mathbf{G} \, \mathbf{H} \, \mathbf{W}_{q}^{-1} \,
	\mathbf{H}^{\top}\mathbf{G}^{\top} + \mu \mathbf{W}_{d}^{-1} \right)^{-1} \: .
	\label{eq:matrix-B-underdetermined}
\end{equation}
Evidently, we have considered that all inverses exist in equations \ref{eq:matrix-B-overdetermined} and \ref{eq:matrix-B-underdetermined}.

The $Q \times D$ matrix $\mathbf{B}$ defined by equation \ref{eq:matrix-B-overdetermined} is commonly used for the case 
in which $D > Q$, i.e., when there are more data than parameters (overdetermined problems).
In this case, we consider that the estimate $\tilde{\mathbf{q}}$ is obtained by solving the following linear system
for $\tilde{\boldsymbol{\delta}}_{q}$ (equation \ref{eq:delta-q-tilde}):
\begin{equation}
	\left( \mathbf{H}^{\top} \mathbf{G}^{\top} \mathbf{W}_{d} \, \mathbf{G} \, \mathbf{H} + 
	\mu \, \mathbf{W}_{q} \right) 
	\tilde{\boldsymbol{\delta}}_{q} = 
	\mathbf{H}^{\top} \mathbf{G}^{\top} \mathbf{W}_{d} \: 
	\boldsymbol{\delta}_{d} \: .
	\label{eq:delta-q-tilde-overdetermined}
\end{equation}
On the other hand, for the cases in which $D < Q$ (underdetermined problems), matrix $\mathbf{B}$ is 
usually defined according to equation \ref{eq:matrix-B-underdetermined}. In this case, the general approach involves 
estimating $\tilde{\mathbf{q}}$ in two steps. The first consists in solving a linear system 
for a dummy vector, which is subsequently used to compute $\tilde{\mathbf{q}}$ by a matrix-vector product as follows:
\begin{equation}
	\begin{split}
		\left( \mathbf{G} \, \mathbf{H} \, \mathbf{W}_{q}^{-1} \,
		\mathbf{H}^{\top}\mathbf{G}^{\top} + \mu \mathbf{W}_{d}^{-1} \right)  
		\mathbf{u} = \boldsymbol{\delta}_{d} \\
		\tilde{\boldsymbol{\delta}}_{q} = \mathbf{W}_{q}^{-1} \, \mathbf{H}^{\top} \mathbf{G}^{\top} \mathbf{u}
	\end{split} \quad ,
	\label{eq:delta-q-tilde-underdetermined}
\end{equation}
where $\mathbf{u}$ is a dummy vector.
After obtaining $\tilde{\boldsymbol{\delta}}_{q}$ (equations \ref{eq:delta-q-tilde-overdetermined} and \ref{eq:delta-q-tilde-underdetermined}),
the estimate $\tilde{\mathbf{q}}$ is computed with equation \ref{eq:delta-q-tilde}.

\subsection{Formulation without reparameterization}
\label{subsec:formulation-without-reparameterization}

Note that, for the particular case in which $\mathbf{H} = \mathbf{I}_{P}$ (equation \ref{eq:reparameterization}), 
where $\mathbf{I}_{P}$ is the identity of order $P$, 
$P = Q$, $\mathbf{p} = \mathbf{q}$, $\bar{\mathbf{p}} = \bar{\mathbf{q}}$ (equation \ref{eq:reparameterization-reference}) and 
$\tilde{\mathbf{p}} = \tilde{\mathbf{q}}$ (equation \ref{eq:vector-p-tilde}).
In this case, the linear system (equations \ref{eq:delta-q-tilde-overdetermined} and \ref{eq:delta-q-tilde-underdetermined}) is directly 
solved for 
\begin{equation}
	\tilde{\boldsymbol{\delta}}_{p} = \tilde{\mathbf{p}} - \bar{\mathbf{p}} \: ,
	\label{eq:delta-p-tilde}
\end{equation}
instead of $\tilde{\boldsymbol{\delta}}_{q}$ (equation \ref{eq:delta-q-tilde}).

\subsection{Linear system solvers}
\label{subsec:linear-system-solvers}

According to their properties, the linear systems associated with over and underdetermined problems 
(equations \ref{eq:delta-q-tilde-overdetermined} and \ref{eq:delta-q-tilde-underdetermined}) can be solved 
by using \textit{direct methods} such as LU, Cholesky or QR factorization, for example \citep[][sections 3.2, 4.2 and 5.2]{golub-vanloan2013}.
These methods involve factorizing the linear system matrix in a product of ``simple" matrices 
(i.e., triangular, diagonal or orthogonal). Here, we consider the \textit{Cholesky factorization}, 
\citep[][p. 163]{golub-vanloan2013}.

Let us consider a real linear system $\mathbf{M} \, \mathbf{x} = \mathbf{y}$, where 
$\mathbf{M}$ is a symmetric and positive definite matrix \citep[][p. 159]{golub-vanloan2013}.
In this case, the Cholesky factorization consists in computing 
\begin{equation}
	\mathbf{M} = \boldsymbol{\mathcal{G}} \boldsymbol{\mathcal{G}}^{\top} \: ,
	\label{eq:Cholesky-factorization}
\end{equation}
where $\boldsymbol{\mathcal{G}}$ is a lower triangular matrix called \textit{Cholesky factor} and having positive diagonal entries.
Given $\boldsymbol{\mathcal{G}}$, the original linear system is replaced by two triangular systems, as follows:
\begin{equation}
	\begin{split}
		\boldsymbol{\mathcal{G}} \, \mathbf{s} &= \mathbf{y} \\
		\boldsymbol{\mathcal{G}}^{\top} \, \mathbf{x} &= \mathbf{s}
	\end{split}
	\label{eq:Cholesky-solver}
\end{equation}
where $\mathbf{s}$ is a dummy vector.
For the overdetermined problem (equation \ref{eq:delta-q-tilde-overdetermined}), 
$\mathbf{M} = \left( \mathbf{H}^{\top} \mathbf{G}^{\top} \mathbf{W}_{d} \, \mathbf{G} \, \mathbf{H} + 
\mu \, \mathbf{W}_{q} \right)$, $\mathbf{x} = \tilde{\boldsymbol{\delta}}_{q}$ and
$\mathbf{y} = \left( \mathbf{H}^{\top} \mathbf{G}^{\top} \mathbf{W}_{d} \: \boldsymbol{\delta}_{d} \right)$.
For the underdetermined problem (equation \ref{eq:delta-q-tilde-underdetermined}),
$\mathbf{M} = \left( \mathbf{G} \, \mathbf{H} \, \mathbf{W}_{q}^{-1} \,
\mathbf{H}^{\top}\mathbf{G}^{\top} + \mu \mathbf{W}_{d}^{-1} \right)$, $\mathbf{x} = \mathbf{u}$ and
$\mathbf{y} = \boldsymbol{\delta}_{d}$ .

The use of direct methods for solving large linear systems may be problematic due to computer 
(i) storage of large matrices and (ii) time to perform matrix operations.
This problem may be specially complicated in equivalent-layer technique for the cases in which 
the sensitivity matrix $\mathbf{G}$ does not have a well-defined structure (sec. \ref{subsec:sensitivity-matrix})

These problems can be overcame by solving the linear system using an iterative method.
These methods produce a sequence of vectors that typically converge to the solution at a
reasonable rate. The main computational cost associated with these methods is usually some matrix-vector products 
per iteration.
The \textit{conjugate gradient} (CG) is a very popular iterative method for solving linear systems in equivalent-layer methods.
This method was originally developed to solve systems having a square and positive definite matrix.
There are two adapted versions of the CG method. The first is called \textit{conjugate gradient normal equation residual} (CGNR) 
\citet[][sec. 11.3]{golub-vanloan2013} or \textit{conjugate gradient least squares} (CGLS) \cite[][p. 165]{aster_etal2019} and is
used to solve overdetermined problems (equation \ref{eq:delta-q-tilde-overdetermined}). 
The second is called \textit{conjugate gradient normal equation error} (CGNE) method 
\citet[][sec. 11.3]{golub-vanloan2013} and is used to solve the underdetermined problems (equation \ref{eq:delta-q-tilde-underdetermined}).
Algorithm \ref{alg:CGLS} outlines the CGLS method applied to the overdetermined problem (equation \ref{eq:delta-q-tilde-overdetermined}).

\section{Floating-point operations}
\label{sec:flops}

Two important factors affecting the efficiency of a given matrix algorithm are the storage and amount of required arithmetic. 
Here, we quantify this last factor associated with different computational strategies to solve the linear system of the
equivalent-layer technique (section \ref{sec:computational-strategies}). To do it, we opted by counting \textit{flops},
which are floating point additions, subtractions, multiplications or divisions \cite[][ p. 12--14]{golub-vanloan2013}.
This is a non-hardware dependent approach that allows us to do direct comparison between different equivalent-layer methods.
Most of the flops count used here can be found in \citet[][p. 12, 106, 107 and 164]{golub-vanloan2013}.

Let us consider the case in which the overdetermined problem
(equation \ref{eq:delta-q-tilde-overdetermined}) is solved by Cholesky factorization (equations \ref{eq:Cholesky-factorization} and \ref{eq:Cholesky-solver})
directly for the parameter vector $\tilde{\mathbf{p}}$ by considering 
the particular case in which $\mathbf{H} = \mathbf{I}_{P}$ (equation \ref{eq:reparameterization} and 
subsection \ref{subsec:formulation-without-reparameterization}),
$\mu = 0$ (equation \ref{eq:function-Gamma}), 
$\mathbf{W}_{d} = \mathbf{I}_{D}$ (equation \ref{eq:function-Phi}) and
$\bar{\mathbf{p}} = \mathbf{0}$ (equation \ref{eq:reparameterization-reference}), 
where $\mathbf{I}_{P}$ and $\mathbf{I}_{D}$ are the identities of order $P$ and $D$, respectively.
Based on the information provided in table \ref{tab:standard-flops}, the total number of flops can be determined by aggregating the flops required for various computations. These computations include the matrix-matrix and matrix-vector products $\mathbf{G}^{\top}\mathbf{G}$ and $\mathbf{G}^{\top}\mathbf{d}$, 
the Cholesky factor $\boldsymbol{\mathcal{G}}$, and the solution of triangular systems. Thus, we can express the total number of flops as follows:
\begin{equation}
	f_{\mathtt{Cholesky}} = \nicefrac{1}{3}D^{3} + 2 D^{2} + 2 \left( P^{2} + P \right) D \: .
	\label{flops:cholesky}
\end{equation}
The same particular overdetermined problem can be solved by using the CGLS method (Algorithm \ref{alg:CGLS}).
In this case, we use table \ref{tab:standard-flops} again to combine the total number of
flops associated with the matrix-vector and inner products defined in line 3, before 
starting the iteration, and the $3$ saxpys, $2$ inner products and $2$ matrix-vector products
per iteration (lines $7-12$). By considering a maximum number of iterations $\mathtt{ITMAX}$, we obtain
\begin{equation}
	f_{\mathtt{CGLS}} = 2P(D+1) + \mathtt{ITMAX} \left[ 2P \left( 2D + 3 \right) + 4D \right] \: .
	\label{flops:cgls}
\end{equation}
The same approach used to deduce equations \ref{flops:cholesky} and \ref{flops:cgls} is
applied to compute the total number of flops for the selected equivalent-layer 
methods discussed in section \ref{sec:computational-strategies}.

%To simplify our analysis, we do not consider the number of flops required to compute the sensitivity matrix 
%$\mathbf{G}$ (equation \ref{eq:predicted-data-vector}) or the auxiliary matrices $\mathbf{A}$, $\mathbf{H}$, $\mathbf{W}_{d}$, $\mathbf{W}_{q}$ 
%(equations \ref{eq:transformation}, \ref{eq:reparameterization}, \ref{eq:function-Phi} and \ref{eq:function-Theta}).
%We also neglect the required flops to compute $\bar{p}$ (equation \ref{eq:reparameterization-reference}), 
%retrieve $\tilde{\mathbf{q}}$ from $\tilde{\boldsymbol{\delta}}_{q}$ (equation \ref{eq:delta-q-tilde}) and computing
%$\boldsymbol{\delta}_{d}$ (equation \ref{eq:delta-d}).

%Figure \ref{fig:1} shows the total flops count for the different methods presented in this review for different number of potential-field data 
%ranging from $10,000$ to $1,000,000$.

\section{Numerical stability}

All equivalent-layer methods aim at obtaining an estimate $\tilde{\mathbf{p}}$ for the parameter vector 
$\mathbf{p}$ (equation \ref{eq:predicted-data-vector}), which contains the physical property of the equivalent sources.
Some methods do it by first obtaining an estimate $\tilde{\mathbf{q}}$ for the reparameterized parameter vector $\mathbf{q}$
(equation \ref{eq:reparameterization}) and then using it to obtain $\tilde{\mathbf{p}}$ (equation \ref{eq:vector-p-tilde}).
The stability of a solution $\tilde{\mathbf{p}}$ against noise in the observed data is rarely addressed.
Here, we follow the numerical stability analysis presented in \citet{siqueira-etal2017}.

For a given equivalent-layer method (section \ref{sec:computational-strategies}), we obtain an estimate 
$\tilde{\mathbf{p}}$ assuming noise-free potential-field data $\mathbf{d}$.
Then, we create $L$ different noise-corrupted data $\mathbf{d}^{\ell}$, $\ell \in \{1:L\}$, by adding 
$L$ different sequences of pseudorandom Gaussian noise to $\mathbf{d}$, all of them having zero mean. 
From each $\mathbf{d}^{\ell}$, we obtain an estimate $\tilde{\mathbf{p}}^{\ell}$. 
Regardless of the particular equivalent-layer method used, the following inequality \citep[][ p. 66]{aster_etal2019} 
holds true:
\begin{equation}
	\Delta p^{\ell} \leq \kappa \; \Delta d^{\ell} \: , \quad \ell \in \{1:L\} \: ,
	\label{eq:condition-number}
\end{equation}
where $\kappa$ is the constant of proportionality between the model perturbation
\begin{equation}
	\Delta p^{\ell} = \frac{\| \tilde{\mathbf{p}}^{\ell} - \tilde{\mathbf{p}} \|}{\| \tilde{\mathbf{p}} \|}
	\: , \quad \ell \in \{1:L\} \: ,
	\label{eq:model-perturbation}
\end{equation}
and the data perturbation
\begin{equation}
	\Delta d^{\ell} = \frac{\| \mathbf{d}^{\ell} - \mathbf{d} \|}{\| \mathbf{d} \|}
	\: , \quad \ell \in \{1:L\} \: ,
	\label{eq:data-perturbation}
\end{equation}
with $\| \cdot \|$ representing the Euclidean norm.
The constant $\kappa$ acts as the condition number associated with the pseudo-inverse in a given linear inversion.
The larger (smaller) the value of $\kappa$, the more unstable (stable) is the estimated solution.
Equation \ref{eq:condition-number} shows a linear relationship between the model perturbation $\Delta p^{\ell}$ and 
the data perturbation $\Delta d^{\ell}$ (equations \ref{eq:model-perturbation} and \ref{eq:data-perturbation}).
We estimate the $\kappa$ (equation \ref{eq:condition-number}) associated with a given 
equivalent-layer method as the slope of the straight line fitted to the $L$ points 
$\left( \Delta p^{\ell} \: , \: \Delta d^{\ell} \right)$.

\section{Notation for subvectors and submatrices}

Here, we use a notation inspired on that presented by \citet[][p. 4]{vanloan1992} to represent subvectors and submatrices.
Subvectors of $\mathbf{d}$, for example, are specified by $\mathbf{d}[\mathbf{i}]$, where $\mathbf{i}$ is a
list of integer numbers that ``pick out'' the elements of $\mathbf{d}$ forming the subvector $\mathbf{d}[\mathbf{i}]$.
For example, $\mathbf{i} = (1, 6, 4, 6)$ gives the subvector $\mathbf{d}[\mathbf{i}] = [ d_{1} \:\: d_{6} \:\: d_{4} \:\: d_{6} ]^{\top} $.
Note that the list $\mathbf{i}$ of indices may be sorted or not and it may also have repeated indices.
For the particular case in which the list has a single element $\mathbf{i} = (i)$, then it can be used to extract the $i$-th element 
$d_{i} \equiv \mathbf{d}[i]$ of $\mathbf{d}$.
Sequential lists can be represented by using the colon notation. We consider two types of sequential lists. The first has starting index is smaller than the final index and increment of $1$. The second has starting index is greater than the final index and increment of $-1$. For example, 
\begin{equation*}
	\begin{split}
		\mathbf{i} = (3:8) &\Leftrightarrow \mathbf{d}[\mathbf{i}] = [ d_{3} \:\: d_{4} \:\: \dots \:\: d_{8} ]^{\top} \\
		\mathbf{i} = (8:3) &\Leftrightarrow \mathbf{d}[\mathbf{i}] = [ d_{8} \:\: d_{7} \:\: \dots \:\: d_{3} ]^{\top} \\
		\mathbf{i} = (:8) &\Leftrightarrow \mathbf{d}[\mathbf{i}] = [ d_{1} \:\: d_{2} \:\: \dots \:\: d_{8} ]^{\top} \\
		\mathbf{i} = (3:) &\Leftrightarrow \mathbf{d}[\mathbf{i}] = [ d_{3} \:\: d_{4} \:\: \dots \:\: d_{D} ]^{\top} \\
	\end{split} \quad ,
\end{equation*}
where $D$ is the number of elements forming $\mathbf{d}$.

The notation above can also be used to define submatrices of a $D \times P$ matrix $\mathbf{G}$. 
For example, $\mathbf{i} = (2, 7, 4, 6)$ and $\mathbf{j} = (1, 3, 8)$ lead to the submatrix
\begin{equation*}
	\mathbf{G}[\mathbf{i}, \mathbf{j}] = \begin{bmatrix}
		g_{21} & g_{23} & g_{28} \\
		g_{71} & g_{73} & g_{78} \\
		g_{41} & g_{43} & g_{48} \\
		g_{61} & g_{63} & g_{68} 
	\end{bmatrix} \: .
\end{equation*}
Note that, in this case, the lists $\mathbf{i}$ and $\mathbf{j}$ ``pick out'', respectively, the rows and columns
of $\mathbf{G}$ that form the submatrix $\mathbf{G}[\mathbf{i}, \mathbf{j}]$.
The $i$-th row of $\mathbf{G}$ is given by the $1 \times P$ vector $\mathbf{G}[i,:]$.
Similarly, the $D \times 1$ vector $\mathbf{G}[:,j]$ represents the $j$-th column.
Finally, we may use the colon notation to define the following submatrix:
\begin{equation*}
	\mathbf{i} = (2:5), \mathbf{j} = (3:7) \Leftrightarrow
	\mathbf{G}[\mathbf{i},\mathbf{j}] = \begin{bmatrix}
		g_{23} & g_{24} & g_{25} & g_{26} & g_{27} \\
		g_{33} & g_{34} & g_{35} & g_{36} & g_{37} \\
		g_{43} & g_{44} & g_{45} & g_{46} & g_{47} \\
		g_{53} & g_{54} & g_{55} & g_{56} & g_{57}
	\end{bmatrix} \: ,
\end{equation*}
which contains the contiguous elements of $\mathbf{G}$ from rows $2$ to $5$ and from columns
$3$ to $7$.

\section{Computational strategies}
\label{sec:computational-strategies}

The linear inverse problem of the equivalent-layer technique (section \ref{sec:linear-inverse-problem}) for the case
in which there are large volumes of potential-field data requires dealing with:
\begin{itemize}
	\item[(i)] the large computer memory to store large and full matrices;
	\item[(ii)] the long computation time to multiply a matrix by a vector; and
	\item[(iii)] the long computation time to solve a large linear system of equations.
\end{itemize}
Here, we review some strategies aiming at reducing the computational cost of the equivalent-layer technique. 
We quantify the computational cost by using flops (section \ref{sec:flops}) and compare the results with those obtained for Cholesky factorization and CGLS (equations \ref{flops:cholesky} and \ref{flops:cgls}). 
We focus on the overall strategies used by the selected methods.

\subsection{Moving window}

The initial approach to enhance the computational efficiency of the equivalent-layer technique 
is commonly denoted \textit{moving window} and involves
first splitting the observed data $d_{i}$, $i \in \{1 : D\}$, into $M$ overlapping subsets (or data windows) 
formed by $D^{m}$ data each, $m \in \{ 1 : M \}$.
The data inside the $m$-th window are usually adjacent to each other and have indices defined by an 
integer list $\mathbf{i}^{m}$ having $D^{m}$ elements.
The number of data $D^{m}$ forming the data windows are not necessarily equal to each other.
Each data window has a $D^{m} \times 1$ observed data vector $\mathbf{d}^{m} \equiv \mathbf{d}[\mathbf{i}^{m}]$.
The second step consists in defining a set of $P$ equivalent sources with scalar physical property $p_{j}$, $j \in \{1:P\}$,
and also split them into $M$ overlapping subsets (or source windows) formed by $P^{m}$ data each, $m \in \{ 1 : M \}$.
The sources inside the $m$-th window have indices defined by an integer list $\mathbf{j}^{m}$ having $P^{m}$ elements.
Each source window has a $P^{m} \times 1$ parameter vector $\mathbf{p}^{m}$ and
is located right below the corresponding $m$-th data window. 
Then, each $\mathbf{d}^{m} \equiv \mathbf{d}[\mathbf{i}^{m}]$ is approximated by 
\begin{equation}
	\mathbf{f}^{m} = \mathbf{G}^{m} \mathbf{p}^{m} \: ,
	\label{eq:predicted-data-window-m}
\end{equation}
where $\mathbf{G}^{m} \equiv \mathbf{G}[\mathbf{i}^{m}, \mathbf{j}^{m}]$ is a submatrix of 
$\mathbf{G}$ (equation \ref{eq:predicted-data-vector}) formed by the elements computed with equation 
\ref{eq:harmonic-function-g-ij} using only the data and equivalent sources located inside the window $m$-th.
The main idea of the moving-window approach is using the $\tilde{\mathbf{p}}^{m}$ estimated for 
each window to obtain (i) an estimate $\tilde{\mathbf{p}}$ of the parameter vector for the entire equivalent layer
or (ii) a given potential-field transformation $\mathbf{t}$ (equation \ref{eq:transformation}).
The main advantages of this approach is that (i) the estimated parameter vector $\tilde{\mathbf{p}}$ or transformed potential field
are not obtained by solving the full, but smaller linear systems and (ii) the full matrix $\mathbf{G}$ (equation \ref{eq:predicted-data-vector})
is never stored.

\cite{leao-silva1989} presented a pioneer work using the moving-window approach.
Their method requires a regularly-spaced grid of observed data on a horizontal plane $z_{0}$. 
The data windows are defined by square local grids of $\sqrt{D'} \times \sqrt{D'}$ adjacent points, all of them having the
same number of points $D'$.
The equivalent sources in the $m$-th data window are located below the observation plane, at a constant vertical distance
$\Delta z_{0}$. They are arranged on a regular grid of $\sqrt{P'} \times \sqrt{P'}$ adjacent points 
following the same grid pattern of the observed data. 
The local grid of sources for all data windows have the same number of elements $P'$.
Besides, they are vertically aligned, but expands the limits of their corresponding data windows,
so that $D' < P'$.
Because of this spatial configuration of observed data and equivalent sources, we have that
$\mathbf{G}^{m} = \mathbf{G}'$ (equation \ref{eq:predicted-data-window-m}) for all data windows 
(i.e., $\forall \: m \in \{1 : M\}$), where $\mathbf{G}'$ is a $D' \times P'$ constant matrix.

By omitting the normalization strategy used by \cite{leao-silva1989}, their method consists in 
directly computing the transformed potential field $t^{m}_{c}$ at the central point $(x^{m}_{c}, y^{m}_{c}, z_{0} + \Delta z_{0})$ 
of each data window as follows:
\begin{equation}
	t^{m}_{c} = \left( \mathbf{a}' \right)^{\top} \mathbf{B}' \: \mathbf{d}^{m} \: , \quad m \in \{ 1 : M \} \: ,
	\label{eq:transformed-field-tmc-LS89}
\end{equation}
where $\mathbf{a}'$ is a $P' \times 1$ vector with elements computed by equation 
\ref{eq:harmonic-function-a-kj} by using all equivalent sources in the $m$-th window and
only the coordinate of the central point in the $m$-th data window and
\begin{equation}
	\mathbf{B}'  = \left( \mathbf{G}' \right)^{\top} 
	\left[ \mathbf{G}' \, \left( \mathbf{G}' \right)^{\top} + \mu \, \mathbf{I}_{D'} \right]^{-1} 
	\label{eq:dummy-matrix-LS89}
\end{equation}
is a particular case of matrix $\mathbf{B}$ associated with underdetermined problems (equation \ref{eq:matrix-B-underdetermined}) 
for the particular case in which $\mathbf{H} = \mathbf{W}_{q} = \mathbf{I}_{P'}$ (equations \ref{eq:reparameterization} and \ref{eq:function-Theta}),
$\mathbf{W}_{d} = \mathbf{I}_{D'}$ (equation \ref{eq:function-Phi}), $\bar{\mathbf{p}} = \mathbf{0}$ 
(equation \ref{eq:reparameterization-reference}), where $\mathbf{I}_{P'}$ and $\mathbf{I}_{D'}$ are identity matrices 
of order $P'$ and $D'$, respectively, and $\mathbf{0}$ is a vector of zeros. 
Due to the presumed spatial configuration of the observed 
data and equivalent sources, $\mathbf{a}'$ and $\mathbf{G}'$ are the same for all data windows.
Hence, only the data vector $\mathbf{d}^{m}$ is modified according to the position of the data window.
Note that equation \ref{eq:transformed-field-tmc-LS89} combines the potential-field transformation
(equation \ref{eq:transformation}) with the solution of the undetermined problem
(equation \ref{eq:delta-q-tilde-underdetermined}).

The method proposed by \cite{leao-silva1989} can be outlined by the Algorithm \ref{alg:LS89}.
Note that \cite{leao-silva1989} directly compute the transformed potential $t^{m}_{c}$ at the central point of
each data window without explicitly computing and storing an estimated for $\mathbf{p}^{m}$ (equation \ref{eq:predicted-data-window-m}).
It means that their method allows computing a single potential-field transformation. 
A different transformation or the same one evaluated at different points require running their moving-data window method again.

The total number of flops in Algorithm \ref{alg:LS89} depends on computing the $P' \times D'$
matrix $\mathbf{B}'$ (equation \ref{eq:dummy-matrix-LS89}) in line 6 and use it to define
the $1 \times P'$ vector $\left(\mathbf{a}' \right)^{\top} \mathbf{B}'$ (line 7) before 
starting the iterations and computing an inner product (equation \ref{eq:transformed-field-tmc-LS89}) per iteration.
We consider that the total number of flops associated with $\mathbf{B}'$ is obtained by 
the matrix-matrix product $\mathbf{G}' \left(\mathbf{G}'\right)^{\top}$, its inverse
and then the premultiplication by $\left(\mathbf{G}'\right)^{\top}$.
By using table \ref{tab:standard-flops} and considering that inverse is computed via  Cholesky factorization, we obtain that the total number of flops for lines $6$ and $7$ is
$2(D')^{2}P' + \nicefrac{7 (D')^{3}}{6} + 2(D')^{2}P'$.
Then, the total number of flops for Algorithm \ref{alg:LS89} is
\begin{equation}
	f_{\mathtt{LS89}} = \nicefrac{7}{6}(D')^{3} + 4P'(D')^{2} + M \, 2 P' \: .
	\label{flops:LS89}
\end{equation}

\cite{soler-uieda2021} generalized the method proposed by \cite{leao-silva1989} for irregularly spaced data on an undulating surface.
A direct consequence of this generalization is that a different submatrix $\mathbf{G}^{m} \equiv \mathbf{G}[\mathbf{i}^{m}, \mathbf{j}^{m}]$ 
(equation \ref{eq:predicted-data-window-m}) must be computed for each window.
Differently from \cite{leao-silva1989}, \cite{soler-uieda2021} store the computed $\tilde{\mathbf{p}}^{m}$ for all windows and
subsequently use them to obtain a desired potential-field transformation (equation \ref{eq:transformation}) as the superposed
effect of all windows.
The estimated $\tilde{\mathbf{p}}^{m}$ for all windows are combined to form a single $P \times 1$ vector $\tilde{\mathbf{p}}$,
which is an estimate for original parameter vector $\mathbf{p}$ (equation \ref{eq:predicted-data-vector}).
For each data window, \cite{soler-uieda2021} solve an overdetermined problem (equation \ref{eq:delta-q-tilde-overdetermined}) 
for $\tilde{\mathbf{p}}^{m}$ by using 
$\mathbf{H} = \mathbf{W}_{q} = \mathbf{I}_{P^{m}}$ (equations \ref{eq:reparameterization} and \ref{eq:function-Theta}),
$\mathbf{W}^{m}_{d}$ (equation \ref{eq:function-Phi}) equal to a diagonal matrix of weights for the data inside the $m$-th window
and $\bar{p} = \mathbf{0}$ (equation \ref{eq:reparameterization-reference}), so that
\begin{equation}
	\left[ \left( \mathbf{G}^{m} \right)^{\top} \mathbf{W}^{m}_{d} \, \mathbf{G}^{m} + 
	\mu \, \mathbf{I}_{P'} \right] 
	\tilde{\mathbf{p}}^{m} = 
	\left( \mathbf{G}^{m} \right)^{\top} \mathbf{W}^{m}_{d} \: 
	\mathbf{d}^{m} \: .
	\label{eq:p-tilde-m-SU21}
\end{equation}

Unlike \cite{leao-silva1989}, \cite{soler-uieda2021} do not adopt a sequential order of the data windows; 
rather, they adopt a randomized order of windows in their  iterations.
The overall steps of the method proposed by \cite{soler-uieda2021} are defined by the Algorithm \ref{alg:SU21}.
For convenience, we have omitted  the details about the randomized window order and the normalization strategy 
employed by \cite{soler-uieda2021}. 
Note that this algorithm starts with a residuals vector $\mathbf{r}$ that is iteratively updated.
The  iterative algorithm in \cite{soler-uieda2021} estimates a solution 
($\tilde{\mathbf{p}}^{m}$ in equation \ref{eq:p-tilde-m-SU21}) using the data and 
the equivalent sources that fall within a moving-data window; however, it calculates the predicted data and 
the residual data in the whole survey data. 
Next, the residual data  that fall within a new position of the data window is used as input data to estimate a new 
solution within the data window which, in turn, is used to calculated a new predicted data and a new residual data 
in the whole survey data.
Regarding the equivalent-source layout, \cite{soler-uieda2021} proposed the block-averaged sources locations 
in which the survey area is divided into horizontal blocks and one single equivalent source is assigned to each block. 
Each single source per block is placed over the layer with its horizontal coordinates given by the average horizontal 
positions of observation points. 
According to \cite{soler-uieda2021}, the block-averaged sources layout may prevent aliasing of the interpolated 
values, specially when the observations are unevenly sampled.
This strategy also reduces the number of equivalent sources without affecting the accuracy of the potential-field interpolation. 
Besides, it reduces the computational load for estimating the physical property on the equivalent layer.
This block-averaged sources layout is not included in the Algorithm \ref{alg:SU21}.

The computational cost of Algorithm \ref{alg:SU21} can be defined in terms of the linear
system (equation \ref{eq:p-tilde-m-SU21}) to be solved for each window (line $10$) and the 
subsequent updates in lines $11$ and $12$.
We consider that the linear system cost can be quantified by the matrix-matrix and 
matrix-vector products $\left(\mathbf{G}^{m}\right)^{\top}\mathbf{G}^{m}$ and
$\left(\mathbf{G}^{m}\right)^{\top}\mathbf{d}^{m}$, respectively, 
and solution of the linear system (line $10$) via Cholesky factorization (equations 
\ref{eq:Cholesky-factorization} and \ref{eq:Cholesky-solver}).
The following updates represent a saxpy without scalar-vector product (line $11$) and a 
matrix-vector product (line 12). In this case, according to table \ref{tab:standard-flops},
the total number of flops associated with Algorithm \ref{alg:SU21} is given by:
\begin{equation}
	f_{\mathtt{SU21}} = M \, 
	\left[ \nicefrac{1}{3}(P')^3 + 2(D'+1)(P')^{2} + (4D'+1)P' \right] \: ,
	\label{flops:SU21}
\end{equation}
where $P'$ and $D'$ represent, respectively, the average number of equivalent sources and 
data at each window.

\subsection{Column-action update}

We call the computational strategy \textit{column-action update} because a single source is used to calculate the predicted data 
and the residual data in the whole survey data.
Hence, a single column of the sensitivity matrix $\mathbf{G}$ (equation \ref{eq:predicted-data-vector}) is
calculated iteratively.

\cite{cordell1992} proposed a computational strategy that was later used by \cite{guspi-novara2009} and relies on
first defining one equivalent source located right below each observed data $d_{i}$, $i \in \{1:D\}$, at a vertical
coordinate $z_{i} + \Delta z_{i}$, where $\Delta z_{i}$ is proportional to the distance from the $i$-th observation point 
$(x_{i}, y_{i}, z_{i})$ to its closest neighbor.
The second step consists in updating the physical property $p_{j}$ of a single equivalent source, $j \in \{1:D\}$ and 
remove its predicted potential field from the observed data vector $\mathbf{d}$, producing a residuals vector $\mathbf{r}$.
At each iteration, the single equivalent source is the one located vertically beneath the observation station of the maximum data residual. 
Next, the predicted data produced by this single source is calculated over all of the observation points and 
a new data residual $\mathbf{r}$ and the $D \times 1$ parameter vector $\mathbf{p}$ containing the physical property 
of all equivalent sources are updated iteratively. 
During each subsequent iteration,  \citeauthor{cordell1992}'s method 
either incorporates a single equivalent source or adjusts an existing equivalent source to match the maximum amplitude 
of the current residual field.
The convergence occurs when all of the residuals are bounded by an envelope of prespecified expected error.
At the end, the algorithm produces an estimate $\tilde{\mathbf{p}}$ for the parameter vector yielding a predicted
potential field $\mathbf{f}$ (equation \ref{eq:predicted-data-vector}) satisfactorily fitting the observed data
$\mathbf{d}$ according to a given criterion.
Note that the method proposed by \cite{cordell1992} iteratively solves the linear $\mathbf{G} \tilde{\mathbf{p}} \approx \mathbf{d}$
with a $D \times D$ matrix $\mathbf{G}$. At each iteration, only a single column of $\mathbf{G}$ (equation \ref{eq:predicted-data-vector}) 
is used.
An advantage of this \textit{column-action update approach} is that the full matrix $\mathbf{G}$ is never stored.

Algorithm \ref{alg:C92} delineates the \citeauthor{cordell1992}'s method.
Note that a single column $\mathbf{G}[:, i_{\mathtt{max}}]$ of the $D \times D$ matrix $\mathbf{G}$ (equation \ref{eq:predicted-data-vector})
is used per iteration, where $i_{\mathtt{max}}$ is the index of the maximum absolute value in $\mathbf{r}$.
As pointed out by \cite{cordell1992}, the method does not necessarily decrease monotonically along the iterations.
Besides, the method may not converge depending on how the vertical distances $\Delta z_{i}$, $i \in \{1:D\}$, 
controlling the depths of the equivalent sources are set.
According to \cite{cordell1992}, the maximum absolute value $r_{\mathtt{max}}$ in $\mathbf{r}$ decreases robustly at the beginning 
and oscillates within a narrowing envelope for the subsequent iterations.

\cite{guspi-novara2009} generalized \citeauthor{cordell1992}'s method to
perform reduction to the pole and other transformations on scattered magnetic observations by using two steps. 
The first step involves computing the vertical component of the observed field using equivalent sources while preserving the magnetization direction.
In the second step, the vertical observation direction is maintained, but the magnetization direction is shifted to the vertical.
The main idea employed by both \cite{cordell1992} and \cite{guspi-novara2009} is an iterative scheme that 
uses a single equivalent source positioned below a measurement station to compute both the predicted data and residual data for all stations. 
This approach entails a computational strategy where a single column of the sensitivity matrix $\mathbf{G}$ (equation \ref{eq:predicted-data-vector}) 
is calculated per iteration.

The total number of flops in Algorithm \ref{alg:C92} consists in 
finding the maximum absolute value in vector $\mathbf{r}$ (line $6$) before the while loop. 
Per iteration, there is a saxpy (line $11$) and 
another search for the maximum absolute value in vector $\mathbf{r}$ (line $12$).
By considering that selecting the maximum absolute value in a $D \times 1$ vector is a $D \log_{2}(D)$ operation \citep[e.g.,][p. 420]{press-etal2007},
the total number of flops in Algorithm \ref{flops:C92} is given by:
\begin{equation}
	f_{\mathtt{C92}} = D \log(D) + \mathtt{ITMAX} \left[2D + D \log_{2}(D) \right] \: .
	\label{flops:C92}
\end{equation}

\subsection{Row-action update}

We call the computational strategy \textit{row-action update} because a single row of the sensitivity matrix  
$\mathbf{G}$ (equation \ref{eq:predicted-data-vector}) is calculated iteratively.
Hence, the equivalent-layer solution is updated by processing a new datum (one matrix row) at each iteration.
To reduce the total processing time and memory usage of equivalent-layer technique, \cite{mendonca-silva1994} proposed 
a strategy called \textit{equivalent data concept}.
The equivalent data concept is grounded on the  principle  that there is a subset of redundant data that does not 
contribute to the final solution and thus can be dispensed.
Conversely, there is a subset of observations, called equivalent data, that  contributes effectively to the 
final solution and fits the remaining observations (redundant data).
Iteratively, \cite{mendonca-silva1994} selected the subset of equivalent data that is substantially smaller than 
the original dataset. 
This selection is carried out by incorporating one data point at a time.

\cite{mendonca-silva1994} proposes an algebraic reconstruction technique (ART) \cite[e.g.,][p. 58]{sluis-vorst1987}
to estimate a parameter vector $\tilde{\mathbf{p}}$ for a regular grid of $P$ equivalent sources on a horizontal plane $z_{0}$.
Such methods iterate on the linear system rows to estimate corrections for the parameter vector,
which may substantially save computer time and memory required to compute and store the full linear system matrix
along the iterations.
The convergence of such \textit{row-update methods} depends on the linear system condition.
The main advantage of such methods is not computing and storing the full linear system matrix, but iteratively using 
its rows.
In contrast to ART-type algorithms, the rows in \cite{mendonca-silva1994} are not processed sequentially. 
Instead, in \cite{mendonca-silva1994}, the rows are introduced according to their residual magnitudes
(maximum absolute value in $\mathbf{r}$), which are computed based on the estimate over the equivalent layer from the previous iteration.
The particular ART method proposed by \cite{mendonca-silva1994} considers that
\begin{equation}
	\mathbf{d} = \begin{bmatrix}
		\mathbf{d}_{e} \\ \mathbf{d}_{r}
	\end{bmatrix} \: , \quad 
	\mathbf{G} = \begin{bmatrix}
		\mathbf{G}_{e} \\ \mathbf{R}_{r}
	\end{bmatrix} \: ,
	\label{eq:partitioned-d-G-MS94}
\end{equation}
where $\mathbf{d}_{e}$ and $\mathbf{d}_{r}$ are $D_{e} \times 1$ and $D_{r} \times 1$ vectors and
$\mathbf{G}_{e}$ and $\mathbf{G}_{r}$ are $D_{e} \times P$ and $D_{r} \times P$ matrices, respectively.
\cite{mendonca-silva1994} designate $\mathbf{d}_{e}$ and $\mathbf{d}_{r}$ as, respectively, \textit{equivalent} and \textit{redundant} data.
With the exception of a normalization strategy, \cite{mendonca-silva1994} calculate a $P \times 1$ estimated parameter vector $\tilde{\mathbf{p}}$ 
by solving an underdetermined problem (equation \ref{eq:delta-q-tilde-underdetermined}) involving only the equivalent data $\mathbf{d}_{e}$ 
(equation \ref{eq:partitioned-d-G-MS94})
for the particular case in which $\mathbf{H} = \mathbf{W}_{p} = \mathbf{I}_{P}$ (equations \ref{eq:reparameterization} and \ref{eq:function-Theta}),
$\mathbf{W}_{d} = \mathbf{I}_{D_{e}}$ (equation \ref{eq:function-Phi}) and $\bar{p} = \mathbf{0}$ (equation \ref{eq:reparameterization-reference}), 
which results in
\begin{equation}
	\begin{split}
		\left(\mathbf{F} + \mu \, \mathbf{I}_{D_{e}} \right) \mathbf{u} = \mathbf{d}_{e} \\
		\tilde{\mathbf{p}} = \mathbf{G}_{e}^{\top} \mathbf{u}
	\end{split} \quad ,
	\label{eq:p-tilde-MS94}
\end{equation}
where $\mathbf{F}$ is a computationally-efficient $D_{e} \times D_{e}$ matrix that approximates $\mathbf{G}_{e} \, \mathbf{G}_{e}^{\top}$.
\cite{mendonca-silva1994} presume that the estimated parameter vector $\tilde{\mathbf{p}}$ obtained from equation \ref{eq:p-tilde-MS94}
leads to a $D_{r} \times 1$ residuals vector
\begin{equation}
	\mathbf{r} = \mathbf{d}_{r} - \mathbf{G}_{r} \tilde{\mathbf{p}} 
	\label{eq:residuals-MS94}
\end{equation}
having a maximum absolute value $r_{\mathtt{max}} \le \epsilon$, where $\epsilon$ is a predefined tolerance.

The overall method of \cite{mendonca-silva1994} is defined by Algorithm \ref{alg:MS94}.
It is important noting that the number $D_{e}$ of equivalent data in $\mathbf{d}_{e}$ increases by one per iteration,
which means that the order of the linear system in equation $\ref{eq:p-tilde-MS94}$ also increases by one at each iteration.
Those authors also propose a computational strategy based on Cholesky factorization \cite[e.g.,][p. 163]{golub-vanloan2013}
for efficiently updating 
$\left(\mathbf{F} + \mu \, \mathbf{I}_{D_{e}} \right)$ at a given iteration (line 16 in Algorithm \ref{alg:MS94}) 
by computing only its new elements with respect to those computed in the previous iteration.

\subsection{Reparameterization}

Another approach for improving the computational performance of equivalent-layer technique consists in 
setting a $P \times Q$ reparameterization matrix $\mathbf{H}$ (equation \ref{eq:reparameterization})
with $Q << P$. 
This strategy has been used in applied geophysics for decades \cite[e.g.,][]{skilling-bryan1984, kennett1988, oldenburg1993, barbosa-etal1997} 
and is known as \textit{subspace method}. 
The main idea relies in reducing the linear system dimension from the original $P$-space to a lower-dimensional subspace (the $Q$-space).
An estimate $\tilde{\mathbf{q}}$ for the reparameterized parameter vector $\mathbf{q}$
is obtained in the $Q$-space and subsequently used to obtain an estimate $\tilde{\mathbf{p}}$ 
for the parameter vector $\mathbf{p}$ (equation \ref{eq:predicted-data-vector}) in the $P$-space by using equation
\ref{eq:reparameterization}. 
Hence, the key aspect of this \textit{reparameterization approach} is solving an appreciably smaller linear inverse problem for 
$\tilde{\mathbf{q}}$ than that for the original parameter vector $\tilde{\mathbf{p}}$ (equation \ref{eq:predicted-data-vector}).

\cite{oliveirajr-etal2013} have used this approach to describe the physical property distribution on the
equivalent layer in terms of piecewise bivariate polynomials.
Specifically, their method consists in splitting a regular grid of equivalent sources into 
source windows inside which the physical-property distribution is described by bivariate polynomial 
functions. The key aspect of their method relies on the fact that the total number of coefficients 
required to define the bivariate polynomials is considerably smaller than the original number of equivalent sources. 
Hence, they formulate a linear inverse problem for estimating the polynomial coefficients and use them later
to compute the physical property distribution on the equivalent layer. 

The method proposed by \cite{oliveirajr-etal2013} consists in solving an overdetermined problem 
(equation \ref{eq:delta-q-tilde-overdetermined}) for estimating the polynomial coefficients 
$\tilde{\mathbf{q}}$ with $\mathbf{W}_{d} = \mathbf{I}_{D}$ (equation \ref{eq:function-Phi}) and
$\bar{q} = \mathbf{0}$ (equation \ref{eq:reparameterization-reference}), so that
\begin{equation}
	\left( \mathbf{H}^{\top} \mathbf{G}^{\top} \mathbf{G} \, \mathbf{H} + 
	\mu \, \mathbf{W}_{q} \right) 
	\tilde{\mathbf{q}} = 
	\mathbf{H}^{\top} \mathbf{G}^{\top} \: \mathbf{d} \: ,
	\label{eq:q-tilde-OBU13}
\end{equation}
where $\mathbf{W}_{q} = \mathbf{H}^{\top}\mathbf{W}_{p} \, \mathbf{H}$ is defined by a matrix $\mathbf{W}_{p}$
representing the zeroth- and first-order Tikhonov regularization \cite[e.g.,][p. 103]{aster_etal2019}.
Note that, in this case, the prior information is defined in the $P$-space for the original parameter vector $\mathbf{p}$
and then transformed to the $Q$-space.
Another characteristic of their method is that it is valid for processing irregularly-spaced data
on an undulating surface.

\cite{mendonca-2020} also proposed a reparameterization approach for the equivalent-layer technique.
Their approach, however, consists in setting $\mathbf{H}$ as a truncated singular value decomposition
(SVD) \cite[e.g.,][p. 55]{aster_etal2019} of the observed potential field. 
Differently from \cite{oliveirajr-etal2013}, however, the method of \cite{mendonca-2020} requires 
a regular grid of potential-field data on horizontal plane.
Another difference is that these authors uses $\mathbf{W}_{q} = \mathbf{I}_{Q}$ (equation \ref{eq:function-Theta}),
which means that the regularization is defined directly in the $Q$-space.

Before \cite{oliveirajr-etal2013} and \cite{mendonca-2020}, \cite{barnes-lumley2011} also proposed a computationally
efficient method for equivalent-layer technique based on reparameterization. 
A key difference, however, is that \cite{barnes-lumley2011} did not set a $P \times Q$ reparameterization matrix $\mathbf{H}$
(equation \ref{eq:reparameterization}) with $Q << P$. 
Instead, they used a matrix $\mathbf{H}$ with $Q \approx 1.7 \, P$. 
Their central idea is setting a reparameterization scheme that groups distant equivalent sources into blocks by
using a bisection process. 
This scheme leads to a quadtree representation of the physical-property distribution on the equivalent layer, 
so that matrix $\mathbf{G}\mathbf{H}$ (equation \ref{eq:predicted-data-vetor-reparameterized}) is notably sparse.
\cite{barnes-lumley2011} explore this sparsity in solving the overdetermined problem for $\tilde{\mathbf{q}}$
(equation \ref{eq:q-tilde-OBU13}) via conjugate-gradient method \cite[e.g.,][sec. 11.3]{golub-vanloan2013}.

We consider an algorithm (not shown) that solves the overdetermined problem (equation \ref{eq:delta-q-tilde-overdetermined}) by
combining the reparameterization with CGLS method (Algorithm \ref{alg:CGLS}).
It starts with a reparameterization step defined by defining a matrix $\mathbf{C} = \mathbf{G \, H}$ 
(equation \ref{eq:predicted-data-vetor-reparameterized}).
Then, the CGLS (Algorithm \ref{alg:CGLS}) is applied by replacing $\mathbf{G}$ with $\mathbf{C}$.
In this case, the linear system is solved by the reparameterized parameter vector $\tilde{\mathbf{q}}$ instead of
$\tilde{\mathbf{p}}$. At the end, the estimated $\tilde{\mathbf{q}}$ is transformed into $\tilde{\mathbf{p}}$ 
(equation \ref{eq:vector-p-tilde}).
Compared to the original CGLS shown in Algorithm \ref{alg:CGLS}, the algorithm discussed here has the additional flops
associated with the matrix-matrix product to compute $\mathbf{C}$ and the matrix-vector product of equation \ref{eq:vector-p-tilde} outside the while loop.
Then, according to table \ref{tab:standard-flops}, the total number of flops given by:
\begin{equation}
	f_{\mathtt{reparam.}} = 2Q(DP+D+1) + 2PQ + \mathtt{ITMAX} \left[ 2Q \left( 2D + 3 \right) + 4D \right] \: .
	\label{flops:reparameterization-cgls}
\end{equation}
The important aspect of this approach is that, for the case in which $Q << P$ (equation \ref{eq:reparameterization}),
the number of flops per iteration can be substantially decreased with respect to those associated with Algorithm \ref{alg:CGLS}.
In this case, the flops decrease per iteration compensates the additional flops required to compute $\mathbf{C}$ and 
obtain $\tilde{\mathbf{p}}$ from $\tilde{\mathbf{q}}$ (equation \ref{eq:vector-p-tilde}).

\subsection{Wavelet compression}

Previously to \cite{barnes-lumley2011}, the idea of transforming the dense matrix $\mathbf{G}$ (equation \ref{eq:predicted-data-vector}) 
into a sparse one has already been used in the context of equivalent-layer technique.
\cite{li-oldenburg2010} proposed a method that applies the discrete wavelet transform to introduce sparsity into 
the original dense matrix $\mathbf{G}$.
Those authors approximate a planar grid of potential-field data by a regularly-spaced grid of equivalent sources,
so that the number of data $D$ and sources $P$ is the same , i.e., $D = P$.
Specifically, \cite{li-oldenburg2010} proposed a method that applies the wavelet transform to the original dense 
matrix $\mathbf{G}$ and sets to zero the small coefficients that are below a given threshold, which results in an 
approximating sparse representation of $\mathbf{G}$ in the wavelet domain.
They first consider the following approximation
\begin{equation}
	\mathbf{d}_{w} \approx \mathbf{G}_{s} \, \mathbf{p}_{w} \: ,
	\label{eq:approximated-linear-system-wavelet-domain}
\end{equation}
where 
\begin{equation}
	\mathbf{d}_{w} = \boldsymbol{\mathcal{W}} \, \mathbf{d} \: , \quad 
	\mathbf{p}_{w} = \boldsymbol{\mathcal{W}} \, \mathbf{p} \: ,
	\label{eq:vectors-dw-pw}
\end{equation}
are the observed data and parameter vector in the wavelet domain; $\boldsymbol{\mathcal{W}}$ is a $D \times D$ orthogonal matrix defining a 
discrete wavelet transform; and $\mathbf{G}_{s}$ is a sparse matrix obtained by setting to zero the elements of
\begin{equation}
	\mathbf{G}_{w} = \boldsymbol{\mathcal{W}} \, \mathbf{G} \, \boldsymbol{\mathcal{W}}^{\top}
	\label{eq:matrix-Gw}
\end{equation}
with absolute value smaller than a given threshold.

\citet{li-oldenburg2010} solve a normalized inverse problem in the wavelet domain.
Specifically, they first define a matrix
\begin{equation}
	\mathbf{G}_{L} = \mathbf{G}_{s} \, \mathbf{L}^{-1}
	\label{eq:matrix-GL}
\end{equation}
and a normalized parameter vector 
\begin{equation}
	\mathbf{p}_{L} = \mathbf{L} \, \mathbf{p}_{w} \: ,
	\label{eq:vector-pL}
\end{equation}
where $\mathbf{L}$ is a diagonal and invertible matrix representing an approximation of the 
first-order Tikhonov regularization in the wavelet domain.
Then they solve an overdetermined problem (equation \ref{eq:delta-q-tilde-overdetermined}) 
to obtain an estimate $\tilde{\mathbf{p}}_{L}$ for $\mathbf{p}_{L}$ (equation \ref{eq:vector-pL}), 
with $\mathbf{G}_{L}$ (equation \ref{eq:matrix-GL}), 
$\mathbf{H} = \mathbf{I}_{P}$ (equations \ref{eq:reparameterization}),
$\mu = 0$ (equation \ref{eq:function-Gamma}), 
$\mathbf{W}_{d} = \mathbf{I}_{D}$ (equation \ref{eq:function-Phi}) and 
$\bar{p} = \mathbf{0}$ (equation \ref{eq:reparameterization-reference}) via 
conjugate-gradient method \cite[e.g.,][sec. 11.3]{golub-vanloan2013}.
Finally, \citet{li-oldenburg2010} compute an estimate $\tilde{\mathbf{p}}$ for the original parameter vector given by
\begin{equation}
	\tilde{\mathbf{p}} = \boldsymbol{\mathcal{W}}^{\top} \left( \mathbf{L}^{-1} \, \tilde{\mathbf{p}}_{L} \right) \: ,
	\label{eq:vector-p-tilde-LO10}
\end{equation}
where the term within parenthesis is an estimate $\tilde{\mathbf{p}}_{w}$ of the parameter vector $\mathbf{p}_{w}$
(equation \ref{eq:vectors-dw-pw}) in the wavelet domain and 
matrix $\boldsymbol{\mathcal{W}}^{\top}$ represents an inverse wavelet transform. 

\subsection{Iterative methods using the full matrix $\mathbf{G}$}

\citet{xia-sprowl1991} introduced an iterative method for estimating the parameter vector $\tilde{\mathbf{p}}$ 
(equation \ref{eq:predicted-data-vector}), which was subsequently adapted to the Fourier domain by \citet{xia-etal1993}.
Their method uses the full and dense sensitivity matrix $\mathbf{G}$ (equation \ref{eq:predicted-data-vector})
(without applying any compression or reparameterization, for example) to compute the predicted data
at all observation points per iteration.
More than two decades later, \cite{siqueira-etal2017} have proposed an iterative method similar to that presented by \citet{xia-sprowl1991}.
The difference is that \citeauthor{siqueira-etal2017}'s algorithm was deduced from 
the \textit{Gauss' theorem} \cite[e.g.,][p. 43]{kellogg1967} and the \textit{total excess of mass} \cite[e.g.,][p. 60]{blakely1996}.
Besides, \citet{siqueira-etal2017} have included a numerical analysis showing that their method produces very stable solutions, 
even for noise-corrupted potential-field data.

The iterative method proposed by \citet{siqueira-etal2017} is outlined in Algorithm \ref{alg:SOB17},
presumes an equivalent layer formed by monopoles (point masses) and can be applied to
irregularly-spaced data on an undulating surface.
Note that the residuals $\mathbf{r}$ are used to compute a correction $\boldsymbol{\Delta}\mathbf{p}$
for the parameter vector at each iteration (line 11), which requires a matrix-vector product involving the 
full matrix $\mathbf{G}$.
Interestingly, this approach for estimating the physical property distribution on an equivalent layer 
is the same originally proposed by \citet{bott1960} for estimating the basement relief under sedimentary basins.
The methods of \citet{xia-sprowl1991} and \citet{siqueira-etal2017} were originally proposed for processing gravity data,
but can be potentially applied to any harmonic function because they actually represent iterative solutions of the 
classical \textit{Dirichlet’s problem} or the \textit{first boundary value problem of potential theory} 
\cite[][ p. 236]{kellogg1967} on a plane.

Recently, \citet{jirigalatu-ebbing2019} presented another iterative method for estimating a parameter 
vector $\tilde{\mathbf{p}}$ (equation \ref{eq:predicted-data-vector}). 
With the purpose of combining different potential-field data, 
their method basically modifies that shown in Algorithm \ref{alg:SOB17} by changing the initial approximation 
and the iterative correction for the parameter vector.
Specifically, \citet{jirigalatu-ebbing2019} replace line $5$ by $\tilde{\mathbf{p}} = \mathbf{0}$, where $\mathbf{0}$ is a vector of zeros,
and line $11$ by $\boldsymbol{\Delta}\mathbf{p} = \omega \, \mathbf{G}^{\top} \mathbf{r}$, where $\omega$ is a positive
scalar defined by trial and error.
Note that this modified approach requires two matrix-vector products involving the full matrix $\mathbf{G}$ per iteration.
To overcome the high computational cost of these two products, \citet{jirigalatu-ebbing2019} set an equivalent layer formed by
prisms and compute their predicted potential field in the wavenumber domain by using the Gauss-FFT technique
\cite{zhao-etal2018}.

The iterative method proposed by \citet{siqueira-etal2017} (Algorithm \ref{alg:SOB17}) requires one entrywise product in line $5$ and a matrix-vector followed by subtraction 
in line $7$ before the while loop. 
At each iteration, there is another entrywise product (line $11$), a half saxpy 
(line $12$) and a saxpy (lines $11$ and $12$). Then, we get from table \ref{tab:standard-flops} that the total number of flops is given by:
\begin{equation}
	f_{\mathtt{SOB17}} = 2D^{2} + 2D + \mathtt{ITMAX} \left( 2D^{2} + 3D \right) \: .
	\label{flops:SOB17}
\end{equation}
Note that the number of flops per iteration in $f_{\mathtt{SOB17}}$ (equation \ref{flops:SOB17}) has the same order of magnitude, but is smaller than that in
$f_{\mathtt{CGLS}}$ (equation \ref{flops:cgls}).

\subsection{Iterative deconvolution}

Recently, \citet{takahashi-etal2020,takahashi-etal2022} proposed the \textit{convolutional equivalent-layer method}, 
which explores the structure of the sensitivity matrix $\mathbf{G}$ (equation \ref{eq:predicted-data-vector}) for 
the particular case in which (i) there is a single equivalent source right below each potential-field
datum and (ii) both data and sources rely on planar and regularly spaced grids.
Specifically, they consider a regular grid of $D$ 
potential-field data at points $(x_{i}, y_{i}, z_{0})$, $i \in \{1:D\}$, on a horizontal plane $z_{0}$.
The data indices $i$ may be ordered along the $x-$ or $y-$direction, which results in an
$x-$ or $y-$oriented grid, respectively.
They also consider a single equivalent source located right below each datum, at a constant vertical coordinate
$z_{0} + \Delta z$, $\Delta z > 0$.
In this case, the number of data and equivalent sources are equal to each other (i.e., $D = P$) and
$\mathbf{G}$ (equation \ref{eq:predicted-data-vector}) assumes a \textit{doubly block Toeplitz} \cite[][p. 28]{jain1989} or 
\textit{block-Toeplitz Toeplitz-block} (BTTB) \cite[][p. 67]{chan-jin2007} structure formed by $N_{B} \times N_{B}$
blocks, where each block has $N_{b} \times N_{b}$ elements, with $D = N_{B} \, N_{b}$.
This particular structure allows formulating the product
of $\mathbf{G}$ and an arbitrary vector as a \textit{fast discrete convolution} via 
\textit{Fast Fourier Transform} (FFT) \cite[][section 4.2]{vanloan1992}.

Consider, for example, the particular case in which $N_{B} = 4$, $N_{b} = 3$ and $D = 12$. In this case,
$\mathbf{G}$ (equation \ref{eq:predicted-data-vector}) is a $12 \times 12$ block matrix given by
\begin{equation}
	\mathbf{G} = \begin{bmatrix}
		\mathbf{G}^{0} & \mathbf{G}^{1} & \mathbf{G}^{2} & \mathbf{G}^{3} \\
		\mathbf{G}^{-1} & \mathbf{G}^{0} & \mathbf{G}^{1} & \mathbf{G}^{2} \\
		\mathbf{G}^{-2} & \mathbf{G}^{-1} & \mathbf{G}^{0} & \mathbf{G}^{1} \\
		\mathbf{G}^{-3} & \mathbf{G}^{-2} & \mathbf{G}^{-1} & \mathbf{G}^{0}
	\end{bmatrix}_{D \times D} \: ,
	\label{eq:matrix-G-BTTB}
\end{equation}
where each block $\mathbf{G}^{n}$, $n \in \{ (1 - N_{B}) : (N_{B} - 1) \}$, is a $3 \times 3$ Toeplitz matrix.
\citet{takahashi-etal2020, takahashi-etal2022} have deduced the specific relationship between blocks $\mathbf{G}^{n}$
and $\mathbf{G}^{-n}$ and also between a given block $\mathbf{G}^{n}$ and its transposed $\left(\mathbf{G}^{n}\right)^{\top}$
according to the harmonic function $g_{ij}$ (equation \ref{eq:harmonic-function-g-ij}) defining the element $ij$ of
the sensitivity matrix $\mathbf{G}$ (equation \ref{eq:predicted-data-vector}) and the orientation of the data grid.

Consider the matrix-vector products
\begin{equation}
	\mathbf{G} \, \mathbf{v} = \mathbf{w}
	\label{eq:product-Gv-w}
\end{equation}
and
\begin{equation}
	\mathbf{G}^{\top} \, \mathbf{v} = \mathbf{w} \: ,
	\label{eq:product-GTv-w}
\end{equation}
involving a $D \times D$ sensitivity matrix $\mathbf{G}$ (equation \ref{eq:predicted-data-vector}) defined in terms of a given
harmonic function $g_{ij}$ (equation \ref{eq:harmonic-function-g-ij}), where 
\begin{equation}
	\mathbf{v} = \begin{bmatrix}
		\mathbf{v}^{0} \\ \vdots \\ \mathbf{v}^{N_{B}-1}
	\end{bmatrix}_{D \times 1} \: , \quad 
	\mathbf{w} = \begin{bmatrix}
		\mathbf{w}^{0} \\ \vdots \\ \mathbf{w}^{N_{B}-1}
	\end{bmatrix}_{D \times 1} \: ,
	\label{eq:vectors-v-w}
\end{equation}
are arbitrary partitioned vectors formed by $N_{B}$ sub-vectors $\mathbf{v}^{n}$ and $\mathbf{w}^{n}$,
$n \in \{ 0 : (N_{B} - 1) \}$, all of them having $N_{b}$ elements.
Equations \ref{eq:product-Gv-w} and \ref{eq:product-GTv-w} can be computed in terms of an auxiliary matrix-vector product
\begin{equation}
	\mathbf{G}_{c} \, \mathbf{v}_{c} = \mathbf{w}_{c} \: ,
	\label{eq:aux-BCCB-system}
\end{equation}
where 
\begin{equation}
	\mathbf{v}_{c} = \begin{bmatrix}
		\mathbf{v}_{c}^{0} \\ \vdots \\ \mathbf{v}_{c}^{N_{B}-1} \\ \mathbf{0}
	\end{bmatrix}_{4D \times 1} \: , \quad 
	\mathbf{w}_{c} = \begin{bmatrix}
		\mathbf{w}_{c}^{0} \\ \vdots \\ \mathbf{w}_{c}^{N_{B}-1} \\ \mathbf{0}
	\end{bmatrix}_{4D \times 1} \: ,
	\label{eq:vectors-vc-wc}
\end{equation}
are partitioned vectors formed by $2N_{b} \times 1$ sub-vectors
\begin{equation}
	\mathbf{v}_{c}^{n} = \begin{bmatrix}
		\mathbf{v}^{n} \\ \mathbf{0}
	\end{bmatrix}_{2N_{b} \times 1} \: , \quad 
	\mathbf{w}_{c}^{n} = \begin{bmatrix}
		\mathbf{w}^{n} \\ \mathbf{0}
	\end{bmatrix}_{2N_{b} \times 1} \: ,
	\label{eq:vectors-vc-wc-ell}
\end{equation}
and $\mathbf{G}_{c}$ is a $4D \times 4D$ \textit{doubly block circulant} \cite[][p. 28]{jain1989} or 
\textit{block-circulant circulant-block} (BCCB) \cite[][p. 76]{chan-jin2007} matrix.
What follows aims at explaining how the original matrix-vector products defined by equations \ref{eq:product-Gv-w}
and \ref{eq:product-GTv-w}, involving a $D \times D$ BTTB matrix $\mathbf{G}$ exemplified by equation \ref{eq:matrix-G-BTTB}, 
can be efficiently computed in terms of the auxiliary matrix-vector product given by equation \ref{eq:aux-BCCB-system}, which
has a $4D \times 4D$ BCCB matrix $\mathbf{G}_{c}$.

Matrix $\mathbf{G}_{c}$ (equation \ref{eq:aux-BCCB-system}) is formed by $2N_{B} \times 2N_{B}$ blocks, 
where each block $\mathbf{G}_{c}^{n}$, $n \in \{(1-N_{B}):(N_{B}-1)\}$ is a $2N_{b} \times 2N_{b}$ circulant matrix.
For the case in which the original matrix-vector product is that defined by equation \ref{eq:product-Gv-w},
the first column of blocks forming the BCCB matrix $\mathbf{G}_{c}$ is given by
\begin{equation}
	\mathbf{G}_{c}[ \, : \, , \, : \, 2N_{b}] = \begin{bmatrix}
		\mathbf{G}_{c}^{0} \\
		\mathbf{G}_{c}^{-1} \\
		\vdots \\
		\mathbf{G}_{c}^{1-N_{B}} \\
		\mathbf{0} \\
		\mathbf{G}_{c}^{N_{B}-1} \\
		\vdots \\
		\mathbf{G}_{c}^{1} \\
	\end{bmatrix}_{4D \times 2N_{b}} \: ,
	\label{eq:matrix-G-BCCB-1st-block-column-G}
\end{equation}
with blocks $\mathbf{G}_{c}^{n}$ having the first column given by
\begin{equation}
	\mathbf{G}_{c}^{n}[ \, : \, , \, 1] = \begin{bmatrix}
		\mathbf{G}^{n}[ \, : \, , \, 1] \\ 
		0 \\ 
		\left( \mathbf{G}^{n}[ 1 \, , \, N_{b}:2] \right)^{\top}
	\end{bmatrix}_{2N_{b} \times 2N_{b}} \: , \quad n \in \{(1-N_{B}):(N_{B}-1)\} \: ,
	\label{eq:block-Gc-ell-1st-column-G}
\end{equation}
where $\mathbf{G}^{n}$ are the blocks forming the BTTB matrix $\mathbf{G}$ (equation \ref{eq:matrix-G-BTTB}).
For the case in which the original matrix-vector product is that defined by equation \ref{eq:product-GTv-w},
the first column of blocks forming the BCCB matrix $\mathbf{G}_{c}$ is given by
\begin{equation}
	\mathbf{G}_{c}[ \, : \, , \, : \, 2N_{b}] = \begin{bmatrix}
		\mathbf{G}_{c}^{0} \\
		\mathbf{G}_{c}^{1} \\
		\vdots \\
		\mathbf{G}_{c}^{N_{B}-1} \\
		\mathbf{0} \\
		\mathbf{G}_{c}^{1-N_{B}} \\
		\vdots \\
		\mathbf{G}_{c}^{-1} \\
	\end{bmatrix}_{4D \times 2N_{b}} \: ,
	\label{eq:matrix-G-BCCB-1st-block-column-GT}
\end{equation}
with blocks $\mathbf{G}_{c}^{n}$ having the first column given by
\begin{equation}
	\mathbf{G}_{c}^{n}[ \, : \, , \, 1] = \begin{bmatrix}
		\left( \mathbf{G}^{n}[ \, 1 \, , \, :] \right)^{\top} \\ 
		0 \\ 
		\mathbf{G}^{n}[ N_{b}:2 \, , \, 1 ]
	\end{bmatrix}_{2N_{b} \times 2N_{b}} \: , \quad n \in \{(1-N_{B}):(N_{B}-1)\} \: .
	\label{eq:block-Gc-ell-1st-column-GT}
\end{equation}
The complete matrix $\mathbf{G}_{c}$ (equation \ref{eq:aux-BCCB-system}) is obtained by properly downshifting 
the block columns $\mathbf{G}_{c}[ \, : \, , \, : \, 2N_{b}]$ defined by equation \ref{eq:matrix-G-BCCB-1st-block-column-G}
or \ref{eq:matrix-G-BCCB-1st-block-column-GT}.
Similarly, the $n$-th block $\mathbf{G}_{c}^{n}$ of $\mathbf{G}_{c}$ is obtained by properly
downshifting the first columns $\mathbf{G}_{c}^{\ell}[ \, : \, , \, 1]$ defined by equation  
\ref{eq:block-Gc-ell-1st-column-G} or \ref{eq:block-Gc-ell-1st-column-GT}.

Note that $\mathbf{G}_{c}$ (equation \ref{eq:aux-BCCB-system}) is a $4D \times 4D$ matrix and 
$\mathbf{G}$ (equation \ref{eq:matrix-G-BTTB}) is a $D \times D$ matrix.
It seems weird to say that computing $\mathbf{G}_{c} \mathbf{v}_{c}$ is more efficient
than directly computing $\mathbf{G} \mathbf{v}$. To understand this, we need first to use the fact that
BCCB matrices are diagonalized by the 2D unitary discrete Fourier transform (DFT) \citep[e.g.,][ p. 31]{davis1979}.
Because of that, $\mathbf{G}_{c}$ can be written as
\begin{equation}
	\mathbf{G}_{c} = 
	\left(\boldsymbol{\mathcal{F}}_{2N_{B}} \otimes \boldsymbol{\mathcal{F}}_{2N_{b}} \right)^{\ast} 
	\boldsymbol{\Lambda}
	\left(\boldsymbol{\mathcal{F}}_{2N_{B}} \otimes \boldsymbol{\mathcal{F}}_{2N_{b}} \right) \: ,
	\label{eq:matrix-G-BCCB-diagonalized}
\end{equation}
where the symbol ``$\otimes$" denotes the Kronecker product \cite[e.g.,][p. 243]{horn-johnson1991},
$\boldsymbol{\mathcal{F}}_{2N_{B}}$ and $\boldsymbol{\mathcal{F}}_{2N_{b}}$ are the $2N_{B} \times 2N_{B}$ and $2N_{b} \times 2N_{b}$ 
unitary DFT matrices \citep[e.g.,][ p. 31]{davis1979}, respectively, the superscritpt 
``$\ast$" denotes the complex conjugate and $\boldsymbol{\Lambda}$ is a 
$4D \times 4D$ diagonal matrix containing the eigenvalues of $\mathbf{G}_{c}$.
Due to the diagonalization of the matrix $\mathbf{G}_{c}$, 
equation \ref{eq:aux-BCCB-system} can be rewritten by using equation 
\ref{eq:matrix-G-BCCB-diagonalized} and premultiplying both sides of the result 
by $\left(\boldsymbol{\mathcal{F}}_{2N_{B}} \otimes \boldsymbol{\mathcal{F}}_{2N_{b}} \right)$, i.e.,
\begin{equation}
	\boldsymbol{\Lambda} \left(\boldsymbol{\mathcal{F}}_{2N_{B}} \otimes \boldsymbol{\mathcal{F}}_{2N_{b}} \right) 
	\mathbf{v}_{c} = \left(\boldsymbol{\mathcal{F}}_{2N_{B}} \otimes \boldsymbol{\mathcal{F}}_{2N_{b}} \right) 
	\mathbf{w}_{c} \: .
	\label{eq:aux-BCCB-system-diagonalized}
\end{equation}
By following \citet{takahashi-etal2020}, we rearrange equation \ref{eq:aux-BCCB-system-diagonalized} as follows
\begin{equation}
	\boldsymbol{\mathcal{L}} \circ 
	\left( \boldsymbol{\mathcal{F}}_{2N_{B}} \, \boldsymbol{\mathcal{V}}_{c} \, \boldsymbol{\mathcal{F}}_{2N_{b}} \right) = 
	\boldsymbol{\mathcal{F}}_{2N_{B}} \, \boldsymbol{\mathcal{W}}_{c} \, \boldsymbol{\mathcal{F}}_{2N_{b}}
	\label{eq:aux-BCCB-system-diagonalized-2}
\end{equation}
where ``$\circ$'' denotes the Hadamard product \cite[e.g.,][p. 298]{horn-johnson1991} and 
$\boldsymbol{\mathcal{L}}$, $\boldsymbol{\mathcal{V}}_{c}$ and $\boldsymbol{\mathcal{W}}_{c}$ are 
$2N_{B} \times 2N_{b}$ matrices obtained 
by rearranging, along their rows, the elements forming the diagonal of 
$\boldsymbol{\Lambda}$ (equation \ref{eq:matrix-G-BCCB-diagonalized}), vector $\mathbf{v}_{c}$ and 
vector $\mathbf{w}_{c}$ (equation \ref{eq:vectors-vc-wc}), respectively.
Then, by premultiplying both sides of equation \ref{eq:aux-BCCB-system-diagonalized-2} 
by $\boldsymbol{\mathcal{F}}_{2N_{B}}^{\ast}$ and then postmultiplying both sides by 
$\boldsymbol{\mathcal{F}}_{2N_{b}}^{\ast}$, we obtain
\begin{equation}
	\boldsymbol{\mathcal{F}}_{2N_{B}}^{\ast} \left[ \boldsymbol{\mathcal{L}} \circ 
	\left( \boldsymbol{\mathcal{F}}_{2N_{B}} \, \boldsymbol{\mathcal{V}}_{c} \, \boldsymbol{\mathcal{F}}_{2N_{b}} \right)
	\right] \boldsymbol{\mathcal{F}}_{2N_{b}}^{\ast} = \boldsymbol{\mathcal{W}}_{c} \: .
	\label{eq:aux-BCCB-system-diagonalized-3}
\end{equation}
Finally, we get from equation \ref{eq:matrix-G-BCCB-diagonalized} that 
matrix $\boldsymbol{\mathcal{L}}$ can be computed by using only the first column 
$\mathbf{G}_{c}[:,1]$ of the BCCB matrix $\mathbf{G}_{c}$ (equation \ref{eq:aux-BCCB-system})
according to \citep{takahashi-etal2020}
\begin{equation}
	\boldsymbol{\mathcal{L}} = \sqrt{4D} \; 
	\boldsymbol{\mathcal{F}}_{2N_{B}} \, \boldsymbol{\mathcal{C}} \, \boldsymbol{\mathcal{F}}_{2N_{b}} \: ,
	\label{eq:matrix-L}
\end{equation}
where $\boldsymbol{\mathcal{C}}$ is a $2N_{B} \times 2N_{b}$ matrix obtained 
by rearranging, along its rows, the elements of $\mathbf{G}_{c}[:,1]$ (equation \ref{eq:aux-BCCB-system}).
It is important noting that the matrices $\boldsymbol{\mathcal{C}}$ and $\boldsymbol{\mathcal{L}}$ (equation \ref{eq:matrix-L}) 
associated with the BTTB matrix $\mathbf{G}$ (equation \ref{eq:matrix-G-BTTB}) are different from those associated with $\mathbf{G}^{\top}$.

The whole procedure to compute the original matrix-vector products $\mathbf{G}\mathbf{v}$ (equation \ref{eq:product-Gv-w})
and $\mathbf{G}^{\top}\mathbf{v}$ (equation \ref{eq:product-GTv-w}) consists in 
(i) rearranging the elements of the vector $\mathbf{v}$ and the first column $\mathbf{G}[:,1]$ of matrix $\mathbf{G}$
into the matrices $\boldsymbol{\mathcal{V}}_{c}$ and $\boldsymbol{\mathcal{C}}$ 
(equations \ref{eq:aux-BCCB-system-diagonalized-3} and \ref{eq:matrix-L}), respectively;
(ii) computing terms $\boldsymbol{\mathcal{F}}_{2N_{B}} \, \boldsymbol{\mathcal{A}} \, \boldsymbol{\mathcal{F}}_{2N_{b}}$
and $\boldsymbol{\mathcal{F}}_{2N_{B}}^{\ast} \, \boldsymbol{\mathcal{A}} \, \boldsymbol{\mathcal{F}}_{2N_{b}}^{\ast}$,
where $\boldsymbol{\mathcal{A}}$ is a given matrix, and a Hadamard product to obtain $\boldsymbol{\mathcal{W}}_{c}$ 
(equation \ref{eq:aux-BCCB-system-diagonalized-3}); and
(iii) retrieve the elements of vector $\mathbf{w}$ (equation \ref{eq:product-Gv-w}) from
$\boldsymbol{\mathcal{W}}_{c}$ (equation \ref{eq:aux-BCCB-system-diagonalized-3}).
It is important noting that the steps (i) and (iii) do not have any computational cost because they involve
only reorganizing elements of vectors and matrices.
Besides, the terms $\boldsymbol{\mathcal{F}}_{2N_{B}} \, \boldsymbol{\mathcal{A}} \, \boldsymbol{\mathcal{F}}_{2N_{b}}$
and $\boldsymbol{\mathcal{F}}_{2N_{B}}^{\ast} \, \boldsymbol{\mathcal{A}} \, \boldsymbol{\mathcal{F}}_{2N_{b}}^{\ast}$
in step (ii) represent, respectively, the 2D Discrete Fourier Transform (2D-DFT) and 
the 2D Inverse Discrete Fourier Transform (2D-IDFT) of $\boldsymbol{\mathcal{A}}$.
These transforms can be efficiently computed by using the 2D Fast Fourier Transform (2D-FFT).
Hence, the original matrix-vector products $\mathbf{G}\mathbf{v}$ (equation \ref{eq:product-Gv-w})
and $\mathbf{G}^{\top}\mathbf{v}$ (equation \ref{eq:product-GTv-w}) can be efficiently computed by using the 2D-FFT.

Algorithms \ref{alg:TOB20-22} and \ref{alg:fast-2D-convolution} show pseudo-codes for the convolutional equivalent-layer 
method proposed by \citet{takahashi-etal2020, takahashi-etal2022}.
Note that those authors formulate the overdetermined problem (equation \ref{eq:delta-q-tilde-overdetermined}) of obtaining 
an estimate $\tilde{\mathbf{p}}$ for the parameter vector $\mathbf{p}$ (equation \ref{eq:predicted-data-vector}) as an \textit{iterative deconvolution} via 
\textit{conjugate gradient normal equation residual} (CGNR) \citet[][sec. 11.3]{golub-vanloan2013} or \textit{conjugate gradient least squares} (CGLS) 
\cite[][p. 165]{aster_etal2019} method. 
They consider $\mathbf{H} = \mathbf{I}_{P}$ (equation \ref{eq:reparameterization}), 
$\mu = 0$ (equation \ref{eq:function-Gamma}), 
$\mathbf{W}_{d} = \mathbf{W}_{q} = \mathbf{I}_{P}$ (equations \ref{eq:function-Phi} and \ref{eq:function-Theta})
and $\bar{\mathbf{p}} = \mathbf{0}$ (equation \ref{eq:reparameterization-reference}).
As shown by \citet{takahashi-etal2020, takahashi-etal2022}, the CGLS produces stable estimates $\tilde{\mathbf{p}}$ for the
parameter vector $\mathbf{p}$ (equation \ref{eq:predicted-data-vector}) in the presence of 
noisy potential-field data $\mathbf{d}$. This is a well-known property of the CGLS method \citep[e.g.,][p. 166]{aster_etal2019}.

The key aspect of Algorithm \ref{alg:TOB20-22} is replacing the matrix-vector products of CGLS (Algorithm \ref{alg:CGLS})
by fast convolutions (Algorithm \ref{alg:fast-2D-convolution}).
A fast convolution requires one 2D-DFT, one 2D-IDFT and an entrywise product of matrices.
We consider that the 2D-DFT/IDFT are computed with 2D-FFT and requires $\kappa \, (4D)\log_{2}(4D)$ flops, where $\kappa = 5$ is compatible with a radix-2 FFT 
\citep[][p. 16]{vanloan1992}, and the entrywise product $24D$ flops because it involves two complex matrices having $4D$ elements
\citep[][p. 36]{golub-vanloan2013}.
Hence, Algorithm \ref{alg:fast-2D-convolution} requires $\kappa \, (16D)\log_{2}(4D) + 26D$ flops,
whereas a conventional matrix-vector multiplication involving a $D \times D$ matrix requires $2D^{2}$ (table \ref{tab:standard-flops}).
Finally, Algorithm \ref{alg:TOB20-22} requires two 2D-FFTs (lines $4$ and $5$), one fast convolution and an inner product (line $8$)
previously to the while loop.
Per iteration, there are three saxpys (lines $12$, $15$ and $16$), two inner products (lines $14$ and $17$) and 
two fast convolutions (lines $13$ and $17$), so that:
\begin{equation}
	f_{\mathtt{CGLS}} = \kappa \, (16D)\log_{2}(4D) + 26D + \mathtt{ITMAX} 
	\left[
	\kappa \, (16D)\log_{2}(4D) + 58D
	\right] \: .
	\label{flops:TOB20}
\end{equation}

\subsection{Direct deconvolution}

The method proposed by \citet{takahashi-etal2020, takahashi-etal2022} can be reformulated to avoid the iterations 
of the conjugate gradient method.
This alternative formulation consists in considering that $\mathbf{v} = \mathbf{p}$ and 
$\mathbf{w} = \mathbf{d}$ in equation \ref{eq:product-Gv-w}, where $\mathbf{p}$ is the parameter vector (equation \ref{eq:predicted-data-vector})
and $\mathbf{d}$ the observed data vector.
In this case, the equality ``$=$" in equation \ref{eq:product-Gv-w} becomes an approximation ``$\approx$".
Then, equation \ref{eq:aux-BCCB-system-diagonalized-2} is manipulated to obtain
\begin{equation}
	\boldsymbol{\mathcal{V}}_{c} \approx 
	\boldsymbol{\mathcal{F}}_{2N_{B}}^{\ast} 
	\left[ \left( \boldsymbol{\mathcal{F}}_{2N_{B}} \, \boldsymbol{\mathcal{W}}_{c} \, \boldsymbol{\mathcal{F}}_{2N_{b}} \right)
	\circ \breve{\boldsymbol{\mathcal{L}}} \right] \boldsymbol{\mathcal{F}}_{2N_{b}}^{\ast} \: ,
	\label{eq:direct-deconvolution}
\end{equation}
where
\begin{equation}
	\breve{\boldsymbol{\mathcal{L}}} = 
	\boldsymbol{\mathcal{L}}^{\ast} \oslash \left( \boldsymbol{\mathcal{L}} \circ \boldsymbol{\mathcal{L}}^{\ast} + \zeta \mathbf{1} \right) \: ,
	\label{eq:matrix-L-Wiener-deconvolution}
\end{equation}
$\mathbf{1}$ is a $4D \times 4D$ matrix of ones, ``$\oslash$" denotes entrywise division and 
$\zeta$ is a positive scalar.
Note that $\zeta = 0$ leads to $\mathbf{1} \oslash \boldsymbol{\mathcal{L}}$.
In this case, the entrywise division may be problematic due to the elements of $\boldsymbol{\mathcal{L}}$ 
having absolute value equal or close to zero. So, a small $\zeta$ is set to avoid this problem
in equation \ref{eq:matrix-L-Wiener-deconvolution}.
Next, we use $\breve{\boldsymbol{\mathcal{L}}}$ to obtain a matrix $\boldsymbol{\mathcal{V}}_{c}$ from equation \ref{eq:direct-deconvolution}.
Finally, the elements of the estimated parameter vector $\tilde{\mathbf{p}}$ are retrieved from the first
quadrant of $\boldsymbol{\mathcal{V}}_{c}$.
This procedure represents a \textit{direct deconvolution} \citep[e.g.,][p. 220]{aster_etal2019}
using a \textit{Wiener filter} \citep[e.g.,][p. 263]{gonzalez-woods2002}.

The required total number of flops associated with the direct deconvolution 
aggregates one 2D-FFT to compute matrix $\boldsymbol{\mathcal{L}}$ (equation \ref{eq:matrix-L}),
one entrywise product $\boldsymbol{\mathcal{L}} \circ \boldsymbol{\mathcal{L}}^{\ast}$ involving complex matrices and
one entrywise division to compute $\breve{\boldsymbol{\mathcal{L}}}$ (equation \ref{eq:matrix-L-Wiener-deconvolution}) and
a fast convolution (Algorithm \ref{alg:fast-2D-convolution}) to evaluate equation \ref{eq:direct-deconvolution},
which results in:
\begin{equation}
	f_{\mathtt{deconv.}} = \kappa \, (12D)\log_{2}(4D) + 72D \: .
	\label{flops:direct-deconv}
\end{equation}
Differently from the convolutional equivalent-layer method proposed by 
\citet{takahashi-etal2020, takahashi-etal2022}, the alternative direct deconvolution presented here 
produces an estimated parameter vector $\tilde{\mathbf{p}}$ directly from the observed data $\mathbf{d}$, in a single step,
avoiding the conjugate gradient iterations.
On the other hand, the alternative method presented here requires estimating a set of tentative parameter vectors $\tilde{\mathbf{p}}$
for different predefined $\zeta$. Besides, there must be criterion to chose the best $\tilde{\mathbf{p}}$ from this tentative set.
This can be made, for example, by using the well-known \textit{L-curve} \citep{hansen1992}.
From a computational point of view, the number of CGLS iterations in the method proposed by \citet{takahashi-etal2020, takahashi-etal2022}
is equivalent to the number of tentative estimated parameter vectors required to form the L-curve in the proposed
direct deconvolution.