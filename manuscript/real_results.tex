\section{Real data application}
\label{sec:real_data}

Gridded data of 1000x500 (500000 observed points) for both grav and mag. Data is at -900m.

Grav equivalent layer depth is 300 m and 50 iterations of the cgls method was used.
Mag equivalent layer depth is 0 m and 200 iterations of the cgls method was used.

On an Intel Core i7 7700HQ@2.8 GHz processor in single processing and single-threading modes
the gravimetric equivalent layer took $9.19$ seconds to estimate the equivalent sources with the convolutional method and $0.51$ seconds with the deconvolutional method. 

The magnetic equivalent layer took $82$ seconds to estimate the equivalent sources with the convolutional method and $0.84$ seconds with the deconvolutional method.

As Carajás area is very large different values of the magnetic main field can be considered. 
The main field declination was calculated using the tool in the website (for the date 01/01/2014): https://www.ngdc.noaa.gov/geomag/calculators/magcalc.shtml
For this application I considered an approximated mid location of the area (latitude $-6.55^{\circ}$ and longitude $-50.75^{\circ})$. The declination is $-19.865^{\circ}$ and the inclination $-7.43915^{\circ}$.
As the source magnetization is unknown inclination and declination equal to the main field is being used for all the equivalent sources.

Gravimetric case:

Means

0.0005096975472675431 (convolutional method)

0.4582999511463665 (deconvolutional method with wiener $\mu = 10^{-22}$)

Standart deviations

0.15492798729938298 (convolutional method)

1.229507199000529 (deconvolutional method with wiener $\mu = 10^{-22}$)
\\\\
Magnetic case:

Means

-0.06404347121632468 (convolutional method)

18.992921718679344 (deconvolutional method with wiener $\mu = 10^{-16}$)

Standart deviations

1.9687559764381535 (convolutional method)

33.641199020925924 (deconvolutional method with wiener $\mu = 10^{-16}$)