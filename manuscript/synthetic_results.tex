\section{Synthetic data simulations}
\label{sec:synthetic_simulations}

For all applications, we generate a model composed by two spheres and a poligonal prism in a regular spaced grid of $50 \times 50$. The upper left sphere has a density contrast of $600 \, kg/m^3$, the right upper sphere a negative contrast of $-500 \, kg/m^3$ and the bottom prism is equal to $550 \, kg/m^3$. To generate the magnetic data, the bodies are in the same position and all of them have the same magnetization intensity and direction ($3.46 \, A/m$ intensity, $35.26^{\circ}$ inclination and $45.0^{\circ}$ declination) within a simulated geomagnetic field direction of $20.0^{\circ}$ inclination and $35.0^{\circ}$ declination. These synthetic datas are shown in figures \ref{fig:4} and \ref{fig:7}, respectively.

\subsection{Stability analysis}

For the stability analysis we show the comparison of the normal equations solution (equation \ref{eq:t_data}) with zeroth-order Tikhonov regularization \citep{aster2018parameter}, the convolutional method for gravimetric and magnetic data (equation \ref{eq:DFT-system_takahashi}), the deconvolutional method (equation \ref{eq:deconvolution_takahashi}) and the deconvolutional method with different values for the Wiener stabilization (equation \ref{eq:wiener_takahashi}). We create $21$ data sets, for both gravity and magnetic data, adding a crescent pseudo-random noise to the original data, which varies from $0\%$ to $10\%$ of the maximum anomaly value in intervals of $0.5\%$. These noises has mean equal to zero and a Gaussian distribution. These synthetic datas are shown in figures \ref{fig:4} and \ref{fig:7}, where panel a of each figure represents the noise free data and panel b is the maximum noised data for gravity and magnetic, respectively.
		
Figure \ref{fig:3} shows how the norms of the predicted equivalent sources varies as the level of the noise is increased for the gravimetric data. We can see that for all methods, a linear tendency can be observed as it is expected. The inclination of the straight line is a indicative of the stability of each method. As show in the graph the deconvolutional method is very unstable and it is really necessary to use a stabilization method to have a good parameter estimative. In contrast, a correct value of the stabilization parameter is necessary to not overshoot the smootheness of the solution as it is the case for the well-known zeroth-order Tikhonov regularization as well. Using this gravimetric data, the optimal value for the Wiener stabilization parameter is $\mu = 10^{-20}$. 

Figure \ref{fig:5} shows the comparison of the predicted data for each method with the original data (figure \ref{fig:4}) using the most noised-corrupted data from the set of the stability analysis. The classical with zeroth-order Tikhonov regularization and the convolutional methods (figures \ref{fig:5}a and \ref{fig:5}b) yield very similar results for the predicted data confirming its similarities with the stabilization despite the bid difference in floating-point operations. Figure \ref{fig:5}c shows the deconvolutional method without a stabilization and demonstrates the necessity to use it for this method. Figure \ref{fig:5}d shows the deconvolutional method with Wiener stabilization $\mu = 10^{-15}$ which is too high, demonstrating the over smoothness of the predicted data. Figures \ref{fig:5}e and \ref{fig:5}f shows the predicted data for an optimal value of the Wiener parameter $\mu = 10^{-20}$ and a low value $\mu = 10^{-25}$, respectively. 

For the magnetic data, figure \ref{fig:6} shows a very similar behavior of the stability as the previous case. The Wiener parameter seems to have the best solution for $\mu = 10^{-13}$. For both types of data the best Wiener parameter seems to be one that produces a low slope for the straight line in the stability analysis, discordant from the classical and convolutional methods. 

Figure \ref{fig:8} shows the comparison of the predicted data for each method with the original magnetic data in figure \ref{fig:7} using the most noised-corrupted data modeled from the stability analyis. As the previous case the classical (figure \ref{fig:8}a) and the convolutional (figure \ref{fig:8}b) methods have very similar predicted data but estimated with less orders of magnitude in floating-point operations. The deconvoutional (figure \ref{fig:8}c) have have a strong disagreement with the observed data showing the nedd for a stabilization method.
Figure \ref{fig:8}d has a value of $\mu = 10^{-10}$ and the predicted data became to smooth by it. The optimal value of the Wiener parameter is shown in figure \ref{fig:8}e with $\mu = 10^{-13}$ and figure \ref{fig:8}f shows a predicted data with a low stablization value with $\mu = 10^{-16}$.


%Gravity synthetic statistics to be included:
%Means
%0.24791339230971493 (Classical method)
%0.25522040542133817 (CG BTTB method)
%0.86010282709889 (Deconvolutional method)
%1.53835137193657 (Deconvolutional w\ Wiener overshoot $\mu$ method)
%0.3134732823974472 (Deconvolutional w\ Wiener optimal $\mu$ method)
%0.5553048046997608 (Deconvolutional w\ Wiener suboptimal $\mu$ method)
%
%Standard deviations
%0.18274083156485463 (Classical method)
%0.18986126212291252 (CG BTTB method)
%1.439293452270024 (Deconvolutional method)
%1.1183051446613188 (Deconvolutional w\ Wiener overshoot $\mu$ method)
%0.2367045031838225 (Deconvolutional w\ Wiener optimal $\mu$ method)
%0.7047326489645682 (Deconvolutional w\ Wiener suboptimal $\mu$ method)
%\\\\
%Magnetic synthetic statistics to be included:
%
%Means
%
%6.572366904728528 (Classical method)
%6.689214592343857 (CG BTTB method)
%971.9310697001104 (Deconvolutional method)
%23.672356138290855 (Deconvolutional w\ Wiener overshoot μ method)
%8.09848247511561 (Deconvolutional w\ Wiener optimal μ method)
%29.264824940161848 (Deconvolutional w\ Wiener suboptimal μ method)
%
%standard deviations
%4.9276131862044315 (Classical method)
%4.9909561696899205 (CG BTTB method)
%1742.0801705255908 (Deconvolutional method)
%17.671099090589646 (Deconvolutional w\ Wiener overshoot μ method)
%6.1358737706945075 (Deconvolutional w\ Wiener optimal μ method)
%29.374157656258866 (Deconvolutional w\ Wiener suboptimal μ method)