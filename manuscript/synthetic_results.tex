\section{Numerical simulations}
\label{sec:numerical-simulations}

\subsection{Flops count}
\label{subsec:flops-count}

Figure \ref{fig:1} shows the total number of flops for solving the overdetermined problem
(equation \ref{eq:delta-q-tilde-overdetermined}) with different equivalent-layer methods
(equations \ref{flops:cholesky}, \ref{flops:cgls}, \ref{flops:LS89}, \ref{flops:SU21}, 
\ref{flops:C92}, \ref{flops:reparameterization-cgls}, \ref{flops:SOB17}, \ref{flops:TOB20},
and \ref{flops:direct-deconv}), by considering 
the particular case in which $\mathbf{H} = \mathbf{I}_{P}$ (equation \ref{eq:reparameterization} and 
subsection \ref{subsec:formulation-without-reparameterization}),
$\mu = 0$ (equation \ref{eq:function-Gamma}), 
$\mathbf{W}_{d} = \mathbf{I}_{D}$ (equation \ref{eq:function-Phi}) and
$\bar{\mathbf{p}} = \mathbf{0}$ (equation \ref{eq:reparameterization-reference}), 
where $\mathbf{I}_{P}$ and $\mathbf{I}_{D}$ are the identities of order $P$ and $D$, respectively.
The flops are computed for different number of potential-field data ranging from $10,000$ to $1,000,000$.
Figure \ref{fig:1} shows that the moving data-window strategy by using 
\citeauthor{leao-silva1989}'s \citeyear{leao-silva1989} method and direct deconvolution are the fastest methods.

The control parameters to run the equivalent-layer methods shown in Figure 
 \ref{fig:1}  are the following: 
i) in CGLS, 
\cite{cordell1992}, 
reparameterization approaches \cite[e.g.,][]{oliveirajr-etal2013; mendonca-2020}, 
\cite{siqueira-etal2017}, and 
\cite{takahashi-etal2020}
(equations \ref{flops:cgls} \ref{flops:C92}), 
\ref{flops:reparameterization-cgls},  \ref{flops:SOB17} and \ref{flops:TOB20}) we set ITMAX = 50; 
ii) in \cite{leao-silva1989} (equation \ref{flops:LS89}) we set 
$D'= 49 \: (7 \times 7)$ and $P' = 225 \: (15 \times 15) $; and
iii) in \cite{soler-uieda2021} (equation  \ref{flops:SU21}) we set 
$D'= P' = 900 \: (30 \times 30)$.

\subsection{Synthetic potential-field data}
\label{subsec:synthetic-data}

We create a model composed of three synthetic bodies:
a sphere centered at $(x, y, z) = (3, -2, 2) \: \mathrm{km}$, with radius of $1 \: \mathrm{km}$;
a sphere centered at $(x, y, z) = (1, 2.5, 1.8) \: \mathrm{km}$, with radius of $750 \: \mathrm{m}$; and
a right prism having polygonal horizontal cross-section, top at $z = 900 \: \mathrm{m}$ and thickness $600 \: \mathrm{m}$.
The density contrasts of the upper left sphere, upper right sphere and prism are, respectively, 
$600 \: \mathrm{kg}/\mathrm{m}^{3}$, $-500 \: \mathrm{kg}/\mathrm{m}^{3}$ and to $550 \: \mathrm{kg}/\mathrm{m}^{3}$. 
All synthetic bodies have a total-magnetization vector with intensity $3.46 \: \mathrm{A}/\mathrm{m}$, inclination $35.26^{\circ}$
and declination $45.0^{\circ}$.
We consider a main geomagnetic field with constant inclination $20.0^{\circ}$ and declination $35.0^{\circ}$.

We have computed noise-free gravity disturbance and total-field anomaly data $\mathbf{d}$ produced by the
model on the same regularly spaced grid of $50 \times 50$ points at $z = 50 \: \mathrm{m}$
(Figures \ref{fig:4}A and \ref{fig:7}A).
We have also simulated additional $L = 20$ gravity data sets $\mathbf{d}^{\ell}$, $\ell \in \{1:L\}$, 
by adding pseudo-random Gaussian noise 
with zero mean and crescent standard deviations to the noise-free data (Figure \ref{fig:1}A).
The standard deviations vary from $0.5\%$ to $10\%$ of the maximum absolute value in the noise-free data.
We applied the same procedure to produce additional $20$ noise-corrupted magnetic data sets from the
noise-free data shown in Figure \ref{fig:7}A.
Figures \ref{fig:4}B and \ref{fig:7}B show, respectively, the gravity disturbance and total-field anomaly data 
corrupted with maximum noise level.
The remaining noise-corrupted gravity and magnetic data are not shown.

\subsection{Stability analysis and upward-continued data}

We set two planar equivalent layers having one source below each datum at a constant vertical coordinate $z = 300 \: \mathrm{m}$.
Note that, in this case, both layers have a number of sources $P$ equal to the number of data $D$.
One layer is formed by point masses and is applied to the synthetic gravity data.
The other is applied to the synthetic magnetic data and is composed of dipoles.

We have applied the Cholesky factorization (equations \ref{eq:Cholesky-factorization} and \ref{eq:Cholesky-solver}), 
the iterative deconvolution (Algorithms \ref{alg:TOB20-22} and \ref{alg:fast-2D-convolution}) proposed by 
\citet{takahashi-etal2020, takahashi-etal2022} and the direct deconvolution (equations \ref{eq:direct-deconvolution} and \ref{eq:matrix-L-Wiener-deconvolution})
with four different values for the parameter $\zeta$ to the $21$ gravity and $21$ magnetic data sets.

For each method, we have obtained one estimate $\tilde{\mathbf{p}}$ from the noise-free gravity data $\mathbf{d}$
and $L=20$ estimates $\tilde{\mathbf{p}}^{\ell}$ from the noise-corrupted gravity data $\mathbf{d}^{\ell}$, $\ell \in \{1:L\}$,
for the planar equivalent layer of point masses, totaling $21$ estimated parameter vectors and 
$20$ pairs $\left( \Delta p^{\ell} \: , \: \Delta d^{\ell} \right)$ of model and data perturbations
(equations \ref{eq:model-perturbation} and \ref{eq:data-perturbation}).
Other $21$ estimates for the parameter vector and 
$20$ pairs $\left( \Delta p^{\ell} \: , \: \Delta d^{\ell} \right)$ were obtained in the same way for the 
equivalent layer of dipoles by using the synthetic magnetic data.
Figures \ref{fig:3} and \ref{fig:6} show the numerical stability curves computed with each method for 
synthetic gravity and magnetic data, respectively.

All these $42$ estimated parameters vectors ($21$ for gravity and $21$ for magnetic data) were obtained by solving 
the overdetermined problem (equation \ref{eq:delta-q-tilde-overdetermined}) with the same method for the particular case in which
$\mathbf{H} = \mathbf{I}$ (equation \ref{eq:reparameterization} and 
subsection \ref{subsec:formulation-without-reparameterization}),
$\mathbf{W}_{d} = \mathbf{W}_{q} = \mathbf{I}$ (equations \ref{eq:function-Phi} and \ref{eq:function-Theta}) and
$\bar{\mathbf{p}} = \mathbf{0}$ (equation \ref{eq:reparameterization-reference}), where $\mathbf{I}$ is the identity of order $D$.

Figure \ref{fig:3} shows how the numerical stability curves vary as the level of the noise is increased for the gravimetric data. We can see that for all methods, a linear tendency is observed as it is expected. The inclination of the straight line indicates the stability of each method. 
As shown in Figure \ref{fig:3}, the deconvolutional method without stabilization parameter (blue dots with $\zeta = 0$ ) exhibits high instability, emphasizing the necessity of using the Wiener filter 
($\zeta > 0$ in equation \ref{eq:matrix-L-Wiener-deconvolution}) to achieve accurate parameter estimation. 
Using these gravimetric data, an optimal stabilization parameter 
$\zeta = 10^{-22}$ (green dots) is necessary to yield a stable solution and prevent overshooting the smootheness of the solution.
For the magnetic data, Figure \ref{fig:6} shows a very similar behavior of the stability as the previous case. 
In this case, the Wiener parameter seems to have the best solution for $\zeta = 10^{-14}$ (green dots). 

For both potential-field data, the estimated slopes (Figures \ref{fig:3} and \ref{fig:6}) obtained by using direct deconvolution with an optimum $\zeta$ (green dots), Cholesky factorization (black dots) and iterative deconvolution (red dots) proposed by \cite{takahashi-etal2020} are close to each other.
This result suggest that the direct deconvolution with an optimum Wiener parameter seems to be one that produces a low slope for the straight line in the stability analysis.

Figures \ref{fig:5} and \ref{fig:8} present the comparison between the residuals due the equivalent layers obtained with the Cholesky factorization, 
iterative deconvolution and direct deconvolution with optimum $\zeta$ using, respectively, gravity and magnetic data (Figures \ref{fig:3} and \ref{fig:6}).
The inverted data are those corrupted with the highest noise level (Figures \ref{fig:4}B and \ref{fig:7}B).

Using the gravimetric data, the Cholesky factorization (Figure \ref{fig:5}A), the iterative deconvolution (Figure \ref{fig:5}B) proposed by \cite{takahashi-etal2020}, and the direct deconvolution with an optimum $\zeta = 10^{-22}$ (Figure \ref{fig:5}E) yield acceptable data fitting despite their significant difference in floating-point operations.
In contrast, the residuals obtained using direct deconvolutional method with a Wiener stabilization  $\zeta$ equal to 
zero (Figure \ref{fig:5}C), 
a high value of $\zeta = 10^{-20}$ (Figure \ref{fig:5}D), and 
a low value of  $\zeta = 10^{-24}$ (Figure \ref{fig:5}F)
result in unacceptable data fitting.

Using the magnetic data, Figure \ref{fig:8} shows a very similar behavior 
as in the previous case, indicating a similar pattern.
Hence, the residuals obtained using the Cholesky factorization 
(Figure \ref{fig:8}A), the iterative deconvolution (Figure \ref{fig:8}B) proposed by \cite{takahashi-etal2020}, and the direct deconvolution with an optimum $\zeta = 10^{-14}$ (Figure \ref{fig:8}E) show acceptable data fittings. 
The remainder residuals (direct deconvolutional method with $\zeta =0$ 
in Figure \ref{fig:8}C, $\zeta = 10^{-12}$ in Figure \ref{fig:8}D, and 
$\zeta = 10^{-16}$ in Figure \ref{fig:8}F) show that the estimated equivalent layers do not fit the noisy magnetic data.

The upward continuation of the potential-field data is a processing technique to predict the data in a higher altitude. 
In practice, the interpreter expects a lower amplitude signal and smoother data as the high frequency anomalies tends to disapear. 
Figure  \ref{fig:grav_up}A shows the true modeled upward gravity data at a height of $-500$ m. 
Figures \ref{fig:grav_up}B--E show the results of the upward processing obtained by using the Cholesky factorization,  the iterative deconvolution proposed by \cite{takahashi-etal2020} and the direct deconvolution with $\zeta = 0$ and $\zeta = 10^{-22}$ (optimum value), respectively.
Except the direct deconvolutional method without stabilization ($\zeta = 0$, Figure \ref{fig:grav_up}D), all the three most stable equivalent-layer methods predict the upward-continued gravity data very reasonable.

Figure \ref{fig:mag_up} shows the comparison between the true
(Figure \ref{fig:mag_up}A)  and the predicted  (Figures \ref{fig:mag_up}B--E) upward-continued magnetic data at a height of $-1400$ m. 
As in the gravimetric case, the predicted upward-continued magnetic data via Cholesky factorization (Figure \ref{fig:mag_up}C),  the iterative deconvolution proposed by \cite{takahashi-etal2020} (Figure \ref{fig:mag_up}D) and the direct deconvolution with optimum value of  $\zeta = 10^{-14}$  
(Figure \ref{fig:mag_up}E) are acceptable, except the direct deconvolutional method without stabilization ($\zeta = 0$, Figure Figure \ref{fig:mag_up}D).

%Gravity synthetic statistics to be included:
%Means
%0.24791339230971493 (Classical method)
%0.25522040542133817 (CG BTTB method)
%0.86010282709889 (Deconvolutional method)
%1.53835137193657 (Deconvolutional w\ Wiener overshoot $\mu$ method)
%0.3134732823974472 (Deconvolutional w\ Wiener optimal $\mu$ method)
%0.5553048046997608 (Deconvolutional w\ Wiener suboptimal $\mu$ method)
%
%Standard deviations
%0.18274083156485463 (Classical method)
%0.18986126212291252 (CG BTTB method)
%1.439293452270024 (Deconvolutional method)
%1.1183051446613188 (Deconvolutional w\ Wiener overshoot $\mu$ method)
%0.2367045031838225 (Deconvolutional w\ Wiener optimal $\mu$ method)
%0.7047326489645682 (Deconvolutional w\ Wiener suboptimal $\mu$ method)
%\\\\
%Magnetic synthetic statistics to be included:
%
%Means
%
%6.572366904728528 (Classical method)
%6.689214592343857 (CG BTTB method)
%971.9310697001104 (Deconvolutional method)
%23.672356138290855 (Deconvolutional w\ Wiener overshoot μ method)
%8.09848247511561 (Deconvolutional w\ Wiener optimal μ method)
%29.264824940161848 (Deconvolutional w\ Wiener suboptimal μ method)
%
%standard deviations
%4.9276131862044315 (Classical method)
%4.9909561696899205 (CG BTTB method)
%1742.0801705255908 (Deconvolutional method)
%17.671099090589646 (Deconvolutional w\ Wiener overshoot μ method)
%6.1358737706945075 (Deconvolutional w\ Wiener optimal μ method)
%29.374157656258866 (Deconvolutional w\ Wiener suboptimal μ method)