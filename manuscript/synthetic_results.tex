\section{Synthetic data simulations}
\label{sec:synthetic_simulations}

For all applications, we generate a model composed by two spheres (PAREI AQUI - ANDRE)

Grav: the prism has a contrast of density equal to $550 kg/m^3$, the left sphere $600 kg/m^3$ and the right sphere $-500 kg/m^3$

Mag: all 3 bodies have magnetization of 3.46 intensity, 35.26 declination and 45.0 inclinations.
The grid of the synthetic data is 50x50.


\subsection{Stability analysis}

For the stability analysis we show the comparison of the normal equations solution with zeroth-order Tikhonov regularization, the convolutional method for gravimetric and magnetic data, the deconvolutional method and the deconvolutional method with different values for the Wiener stabilization. We create $21$ data sets adding a crescent pseudo-random noise to the original data, which varies from $0\%$ to $10\%$ of the maximum anomaly value, in intervals of $0.5\%$. These noises has mean equal to zero and a Gaussian distribution.
Figure XX shows how the residual between the predicted data and the noise-free data changes as the level of the noise is increased. We can see that for all methods, a linear tendency can be observed as it is expected. The inclination of the straight line is a indicative of the stability of each method. As show in the graph the deconvolutional method is very unstable and it is really necessary to use a stabilization method to have a good parameter estimative. In contrast, a correct value of the stabilization parameter is necessary to not overshoot the smootheness of the solution as it is the case for the well-known zeroth-order Tikhonov regularization. For the example using this gravimetric data, the optimal value for the Wiener stabilization parameter is $\mu = 10^{-9}$. Figure XX shows the comparison of the predicted data for each method with the original data.

For the magnetic data, the Wiener parameter seems to have the best solution for $\mu = 10^{-13}$. Figure XX shows the comparison of the predicted data for each method with the original data.


Gravity synthetic statistics to be included:

Means

0.24791339230971493 (Classical method)

0.25522040542133817 (CG BTTB method)

0.86010282709889 (Deconvolutional method)

1.53835137193657 (Deconvolutional w\ Wiener overshoot $\mu$ method)

0.3134732823974472 (Deconvolutional w\ Wiener optimal $\mu$ method)

0.5553048046997608 (Deconvolutional w\ Wiener suboptimal $\mu$ method)

Standard deviations

0.18274083156485463 (Classical method)

0.18986126212291252 (CG BTTB method)

1.439293452270024 (Deconvolutional method)

1.1183051446613188 (Deconvolutional w\ Wiener overshoot $\mu$ method)

0.2367045031838225 (Deconvolutional w\ Wiener optimal $\mu$ method)

0.7047326489645682 (Deconvolutional w\ Wiener suboptimal $\mu$ method)
\\\\
Magnetic synthetic statistics to be included:

Means

4.71700586977892 (Classical method)

4.791561788289162 (CG BTTB method)

2761.170136908849 (Deconvolutional method)

23.06442216727077 (Deconvolutional w\ Wiener overshoot $\mu$ method)

8.860326967314595 (Deconvolutional w\ Wiener optimal $\mu$ method)

33.83142137645327 (Deconvolutional w\ Wiener suboptimal $\mu$ method)

Standard deviations

3.571868619217408 (Classical method)

3.6420538498453263 (CG BTTB method)

4507.745135864884 (Deconvolutional method)

14.232818621809827 (Deconvolutional w\ Wiener overshoot $\mu$ method)

7.445085208152109 (Deconvolutional w\ Wiener optimal $\mu$ method)

47.8175518946965 (Deconvolutional w\ Wiener suboptimal $\mu$ method)